\documentclass[a4paper,11pt]{book}
%\documentclass[a4paper,twoside,11pt,titlepage]{book}
\usepackage{listings}
\usepackage[utf8]{inputenc}
\usepackage[spanish]{babel}

%% Allows 'H' property in images, which makes possible
%% to force an image into a certain position
\usepackage{float}

%% In order to crop images
\usepackage{graphicx}

\usepackage[nottoc]{tocbibind}

\decimalpoint
\usepackage{dcolumn}
\newcolumntype{.}{D{.}{\esperiod}{-1}}
\makeatletter
\addto\shorthandsspanish{\let\esperiod\es@period@code}
\makeatother


%\usepackage[chapter]{algorithm}
\RequirePackage{verbatim}
%\RequirePackage[Glenn]{fncychap}
\usepackage{fancyhdr}
\usepackage{graphicx}
\usepackage{afterpage}

\usepackage{longtable}

\usepackage[hidelinks=true,pdfborder={000}]{hyperref} %referencia
\newcommand{\MYhref}[3][blue]{\href{#2}{\color{#1}{#3}}}%

% ********************************************************************
% Re-usable information
% ********************************************************************
\newcommand{\myTitle}{Desarrollo de un videojuego para la configuración y análisis de redes de computadores\xspace}
\newcommand{\myDegree}{Grado en Ingeniería de Tecnologías de Telecomunicación\xspace}
\newcommand{\myName}{Ángel Oreste Rodríguez Romero\xspace}
\newcommand{\myProf}{Juan José Ramos Muñoz\xspace}
\newcommand{\myOtherProf}{Jonathan Prados Garzón\xspace}
%\newcommand{\mySupervisor}{Put name here\xspace}
\newcommand{\myFaculty}{Escuela Técnica Superior de Ingenierías Informática y de
Telecomunicación\xspace}
\newcommand{\myFacultyShort}{E.T.S. de Ingenierías Informática y de
Telecomunicación\xspace}
\newcommand{\myDepartment}{Departamento de TSTC\xspace}
\newcommand{\myUni}{\protect{Universidad de Granada}\xspace}
\newcommand{\myLocation}{Granada\xspace}
\newcommand{\myTime}{\today\xspace}
\newcommand{\myVersion}{Version 0.1\xspace}

%% My commands
\newcommand{\GNSCS}{\texttt{GNS3sharp}}
\newcommand{\NODE}{\texttt{Node}}
\newcommand{\LINK}{\texttt{Link}}
\newcommand{\GAOBJ}{\texttt{GameObject}}


\hypersetup{
pdfauthor = {\myName (aoreste96@correo.ugr.es)},
pdftitle = {\myTitle},
pdfsubject = {},
pdfkeywords = {GNS3, Unity, telemática, API, ...},
pdfcreator = {LaTeX con el paquete ....},
pdfproducer = {pdflatex}
}

%\hyphenation{}


%\usepackage{doxygen/doxygen}
%\usepackage{pdfpages}
\usepackage{url}
\usepackage{colortbl,longtable}
\usepackage[stable]{footmisc}
%\usepackage{index}

%\makeindex
%\usepackage[style=long, cols=2,border=plain,toc=true,number=none]{glossary}
% \makeglossary

% Definición de comandos que me son tiles:
%\renewcommand{\indexname}{Índice alfabético}
%\renewcommand{\glossaryname}{Glosario}

\pagestyle{fancy}
\fancyhf{}
\fancyhead[LO]{\leftmark}
\fancyhead[RE]{\rightmark}
\fancyhead[RO,LE]{\textbf{\thepage}}
\renewcommand{\chaptermark}[1]{\markboth{\textbf{#1}}{}}
\renewcommand{\sectionmark}[1]{\markright{\textbf{\thesection. #1}}}

\setlength{\headheight}{1.5\headheight}

\newcommand{\HRule}{\rule{\linewidth}{0.5mm}}
%Definimos los tipos teorema, ejemplo y definición podremos usar estos tipos
%simplemente poniendo \begin{teorema} \end{teorema} ...
\newtheorem{teorema}{Teorema}[chapter]
\newtheorem{ejemplo}{Ejemplo}[chapter]
\newtheorem{definicion}{Definición}[chapter]

\definecolor{gray97}{gray}{.97}
\definecolor{gray75}{gray}{.75}
\definecolor{gray45}{gray}{.45}
\definecolor{gray30}{gray}{.94}

\lstset{ frame=Ltb,
     framerule=0.5pt,
     aboveskip=0.5cm,
     framextopmargin=3pt,
     framexbottommargin=3pt,
     framexleftmargin=0.1cm,
     framesep=0pt,
     rulesep=.4pt,
     backgroundcolor=\color{gray97},
     rulesepcolor=\color{black},
     %
     stringstyle=\ttfamily,
     showstringspaces = false,
     basicstyle=\scriptsize\ttfamily,
     commentstyle=\color{gray45},
     keywordstyle=\bfseries,
     %
     numbers=left,
     numbersep=6pt,
     numberstyle=\tiny,
     numberfirstline = false,
     breaklines=true,
   }
 
% minimizar fragmentado de listados
\lstnewenvironment{listing}[1][]
   {\lstset{#1}\pagebreak[0]}{\pagebreak[0]}

\renewcommand{\lstlistingname}{Código}
\lstloadlanguages{C,C++,csh,Java}

\definecolor{red}{rgb}{0.6,0,0} 
\definecolor{blue}{rgb}{0,0,0.6}
\definecolor{green}{rgb}{0,0.8,0}
\definecolor{cyan}{rgb}{0.0,0.6,0.6}
\definecolor{cloudwhite}{rgb}{0.9529,0.949,0.9056}

\lstset{
	language=csh,
	basicstyle=\footnotesize\ttfamily,
	numbers=none,
	numberstyle=\tiny,
	numbersep=5pt,
	tabsize=2,
	extendedchars=true,
	breaklines=true,
	frame=b,
	stringstyle=\color{blue}\ttfamily,
	showspaces=false,
	showtabs=false,
	xleftmargin=17pt,
	framexleftmargin=17pt,
	framexrightmargin=5pt,
	framexbottommargin=4pt,
	commentstyle=\color{green},
	morecomment=[l]{//}, %use comment-line-style!
	morecomment=[s]{/*}{*/}, %for multiline comments
	showstringspaces=false,
	morekeywords={ abstract, event, new, struct,
	as, explicit, null, switch,
	base, extern, object, this,
	bool, false, operator, throw,
	break, finally, out, true,
	byte, fixed, override, try,
	case, float, params, typeof,
	catch, for, private, uint,
	char, foreach, protected, ulong,
	checked, goto, public, unchecked,
	class, if, readonly, unsafe,
	const, implicit, ref, ushort,
	continue, in, return, using,
	decimal, int, sbyte, virtual,
	default, interface, sealed, volatile,
	delegate, internal, short, void,
	do, is, sizeof, while,
	double, lock, stackalloc,
	else, long, static,
	enum, namespace, string},
	keywordstyle=\color{cyan},
	identifierstyle=\color{red},
	backgroundcolor=\color{cloudwhite},
}
  


\newcommand{\bigrule}{\titlerule[0.5mm]}


%Para conseguir que en las páginas en blanco no ponga cabecerass
\makeatletter
\def\clearpage{%
  \ifvmode
    \ifnum \@dbltopnum =\m@ne
      \ifdim \pagetotal <\topskip
        \hbox{}
      \fi
    \fi
  \fi
  \newpage
  \thispagestyle{empty}
  \write\m@ne{}
  \vbox{}
  \penalty -\@Mi
}
\makeatother

\usepackage{pdfpages}

%%%%%%%%%%%%%%%%%%%%%%%%%%%%%%%%%%%%%%%%%%%%
\begin{document}
% 0. Portada y prefacio
\begin{titlepage}
 
 
\newlength{\centeroffset}
\setlength{\centeroffset}{-0.5\oddsidemargin}
\addtolength{\centeroffset}{0.5\evensidemargin}
\thispagestyle{empty}

\noindent\hspace*{\centeroffset}\begin{minipage}{\textwidth}

\centering
\includegraphics[width=0.9\textwidth]{imagenes/logo_ugr.jpg}\\[1.4cm]

\textsc{ \Large TRABAJO FIN DE GRADO\\[0.2cm]}
\textsc{INGENIERÍA DE TECNOLOGÍAS DE TELECOMUNICACIÓN}\\[1cm]
% Upper part of the page
% 
% Title
{\Large\bfseries Desarrollo de un videojuego para la configuración y análisis de redes de computadores\\
}
\noindent\rule[-1ex]{\textwidth}{3pt}\\[3.5ex]
{\large\bfseries GNS3sharp}
\end{minipage}

\vspace{2.5cm}
\noindent\hspace*{\centeroffset}\begin{minipage}{\textwidth}
\centering

\textbf{Autor}\\ {Ángel Oreste Rodríguez Romero}\\[2.5ex]
\textbf{Directores}\\
{Juan José Ramos Muñoz\\
Jonathan Prados Garzón}\\[2cm]
\includegraphics[width=0.3\textwidth]{imagenes/etsiit_logo.png}\\[0.1cm]
\textsc{Escuela Técnica Superior de Ingenierías Informática y de Telecomunicación}\\
\textsc{---}\\
Granada, agosto de 2018
\end{minipage}
%\addtolength{\textwidth}{\centeroffset}
%\vspace{\stretch{2}}
\end{titlepage}



\chapter*{}
%\thispagestyle{empty}
%\cleardoublepage

%\thispagestyle{empty}

\input{portada/portada_2}



\cleardoublepage
\thispagestyle{empty}

\begin{center}
{\large\bfseries Desarrollo de un videojuego para la configuración y análisis de redes de computadores: GNS3sharp}\\
\end{center}
\begin{center}
Ángel Oreste Rodríguez Romero\\
\end{center}

%\vspace{0.7cm}
\noindent{\textbf{Palabras clave}: palabra\_clave1, palabra\_clave2, palabra\_clave3, ......}\\

\vspace{0.7cm}
\noindent{\textbf{Resumen}}\\

El jugar ha sido desde siempre un gran amigo de la educación. Aprender jugando es un lema que cada vez se repite más. Los videojuegos, concretamente, toman en cierta forma el relevo de los juegos tal y como tradicionalmente estos se han entendido y amplían sus posibilidades.

La era digital afecta a casi todos los ámbitos de nuestro entorno. Las redes no iban a quedar excluidas de ese avance. Así, se pueden encontrar decenas de implementaciones virtuales de redes de telecomunicaciones, permitiéndonos visualizar su funcionamiento evitando el desembolso que equivale una real.

Digitalizados ambos ámbitos, ¿por qué no unirlos? ¿Y por qué no unirlos con un propósito educacional?

El presente documento pretende realizar un acercamiento a tal propósito. Se listará una serie de tecnologías que permiten llevar esto a cabo así como el desarrollo de mi aproximación.
\cleardoublepage


\thispagestyle{empty}


\begin{center}
{\large\bfseries Development of a videogame for the configuration and analysis of computer networks: GNS3sharp}\\
\end{center}
\begin{center}
Ángel Oreste, Rodríguez Romero\\
\end{center}

%\vspace{0.7cm}
\noindent{\textbf{Keywords}: Keyword1, Keyword2, Keyword3, ....}\\

\vspace{0.7cm}
\noindent{\textbf{Abstract}}\\

Playing has always been a great friend of education. Learning by playing is a motto that is repeated more and more. Videogames, in particular, take over from games as they have traditionally been understood and expand their possibilities.

The digital age affects almost every area of our environment. Networks would not be excluded from this development. Thus, dozens of virtual implementations of telecommunication networks can be found, allowing us to visualize them working, avoiding the disbursement that is equivalent to a real one.

Digitized both areas, why not unite them, and why not unite them for an educational purpose?

This document is intended to bring this about. A number of technologies will be listed that allow this to be done as well as the development of my approach.

\chapter*{}
\thispagestyle{empty}

\noindent\rule[-1ex]{\textwidth}{2pt}\\[4.5ex]

Yo, \textbf{Ángel Oreste Rodríguez Romero}, alumno de la titulación Ingeniería de Tecnologías de Telecomunicación de la \textbf{Escuela Técnica Superior
de Ingenierías Informática y de Telecomunicación de la Universidad de Granada}, con DNI 25351379C, autorizo la
ubicación de la siguiente copia de mi Trabajo Fin de Grado en la biblioteca del centro para que pueda ser
consultada por las personas que lo deseen.

\vspace{6cm}

\noindent Fdo: Ángel Oreste Rodríguez Romero

\vspace{2cm}

\begin{flushright}
Granada a 1 de septiembre de 2018.
\end{flushright}


\chapter*{}
\thispagestyle{empty}

\noindent\rule[-1ex]{\textwidth}{2pt}\\[4.5ex]

D. \textbf{Juan José Ramos Muñoz}, Profesor del Área de Telemática del Departamento TSTC de la Universidad de Granada.

\vspace{0.5cm}

D. \textbf{Jonathan Prados Garzón}, Profesor del Área de Telemática del Departamento TSTC de la Universidad de Granada.


\vspace{0.5cm}

\textbf{Informan:}

\vspace{0.5cm}

Que el presente trabajo, titulado \textit{\textbf{Desarrollo de un videojuego para la configuración y análisis de redes de computadores: GNS3sharp}},
ha sido realizado bajo su supervisión por \textbf{Ángel Oreste Rodríguez Romero}, y autorizamos la defensa de dicho trabajo ante el tribunal
que corresponda.

\vspace{0.5cm}

Y para que conste, expiden y firman el presente informe en Granada a 1 de septiembre de 2018.

\vspace{1cm}

\textbf{Los directores:}

\vspace{5cm}

\noindent \textbf{Juan José Ramos Muñoz \ \ \ \ \ Jonathan Prados Garzón}

\chapter*{Agradecimientos}
\thispagestyle{empty}

       \vspace{1cm}

A Juanjo, por su increíble paciencia e inestimable ayuda, tanto en nuestras tutorías presenciales como aquellas improvisadas por Telegram. A todos aquellos profesores que confiaron en mí y en mis capacidades más que yo mismo en tantos momentos. A todos aquellos compañeros como Alfonso que no solo me facilitaron la vida académica con su conocimiento, si no también por su compañía. A Alberto, que aunque algo reticente de primeras, está dispuesto a echarme una mano de pedírselo. A mis compañeros de Francia, que me impulsaron a ampliar mis conocimientos. A Antonio por el logo tan genial que ha hecho. A mi grupo, porque sin él no habría tenido el ánimo suficiente durante este año para afrontar el proyecto. A todos aquellos amigos que me oyeron quejarme de mi proyecto con estoicismo. A la comunidad de StackOverflow que de tantos apuros me ha sacado.

Pero ante todo, a mis padres, pilar fundamental y soporte absoluto de toda mi vida, universitaria o no. Por la educación que me han dado, por todos aquellos caprichos que me permitieron y por velar siempre por mi salud.

% 0. Tablas de contenido
\frontmatter
\tableofcontents
\listoffigures
%\listoftables

% Capítulos
\mainmatter
\setlength{\parskip}{5pt}
\chapter{Introducción}\label{chap:Intro}
Este primer capítulo introductorio sirve de acercamiento tanto al problema con el que se pretende resolver como al modo con el que se decidió resolverlo. Al final del mismo se listarán los distintos capítulos del documento así como un breve resumen de cada uno de ellos.

\section{Introducción}
El siguiente trabajo se propone desarrollar un \textbf{videojuego capaz de integrar un simulador de redes} en sus mecánicas; esto es, capaz de \textbf{verse afectado} por la configuración establecida en una red virtualizada y, a la misma vez, tener el poder de modificarla desde el propio juego. La finalidad de todo esto no reside en el entretenimiento puro que brinda el mundo del videojuego, sino en las puertas que abre para el mundo docente. Si consideramos a los videojuegos un medio para facilitar el aprendizaje, de hacerlo más atractivo al público, la integración de redes reales en ellos puede permitir crear un nuevo camino docente en el ámbito de las redes. Finalmente no es más que una forma de facilitar el aprendizaje para el mayor número de personas posible.

Se analizará una serie de tecnologías para seleccionar las más adecuadas para el trabajo. Como resultado de todo lo anterior, se proporcionará una biblioteca para que los desarrolladores puedan crear sus propios juegos sobre el simulador de red, y un juego de ejemplo de uso.

\subsection{Motivación}
\subsubsection{Los videojuegos y la educación}
Contrariamente al modelo educativo tradicional, en que el que la atención gira en torno al profesor, desde principios del nuevo milenio aparece el \textit{aprendizaje centrado en el estudiante}. Aunque el origen del concepto puede datarse a principios del siglo pasado, no ha sido hasta este que ha comenzado a tomar forma. Carl Rogers, en su libro ``Freedom to Learn for the 80s'', describe el cambio de poder del profesor experto al alumno, impulsado por la necesidad de un cambio en el entorno tradicional en el que, en esta "llamada atmósfera educativa, los estudiantes se vuelven pasivos, apáticos y aburridos"\cite{studentcentered}.

R.M Harden y Joy Crosby, en un artículo realizado en el 2000, describen las estrategias de aprendizaje centradas en el ``maestro'' desde el enfoque de \textit{el maestro que transmite el conocimiento}, del experto al principiante. En contraste, describen el aprendizaje centrado en el estudiante como aquel enfocado en el aprendizaje de los estudiantes y ``lo que los estudiantes hacen para lograrlo, en lugar de lo que hace el maestro''. Esta definición enfatiza el concepto del estudiante \textit{haciendo} \cite{goodteacher}.

Es aquí donde los videojuegos pueden ayudar. Este mismo año, un informe salía a la luz dando una serie de datos bastante interesantes al respecto. El informe, desarrollado por \href{https://triseum.com/}{Triseum} (empresa dedicada de lleno a los videojuegos educativos), es un proyecto europeo con el propósito de probar y evaluar el aprendizaje basado en juegos con dos videojuegos educativos (ambos juegos creados por la propia empresa del estudio). De entre los 20 profesores que llevaron a sus aulas los juegos, ninguno de ellos tuvo experiencias negativas con ellos. Algunos de los más satisfechos con la prueba contaban que ``los estudiantes estaban muy motivados y mucho mejor preparados para las siguientes asignaturas y conceptos de la clase, construidos mediante el contenido y los conceptos tratados en el juego''\cite{triseum}.

Cientos de estudios se han llevado a cabo con el fin de demostrar el potencial didáctico de los videojuegos. Algunos de ellos se han centrado en exponer el uso de juegos comerciales en escuelas \cite{gamesinclass}, y otros tantos  el de juegos nacidos por y para la educación \cite{edugamesinclass}. Todos ellos, como el realizado por Kennedy-Clark y Thompson en 2011, concluyen que los beneficios de usar los videojuegos como método de aprendizaje funciona. Este último, asevera que ``como se encontró en muchos otros estudios, el mundo virtual fue motivador, interesante y atractivo para los estudiantes''\cite{deathrome}.

Y es que la virtualización es una herramienta enormemente versátil. Empresas como Labster se han dado cuenta de ello. Esta compañía se encarga de desarrollar laboratorios virtuales y propuestas lúdicas para la enseñanza de materias relacionadas con la ciencia, muchas de ellas llevadas a cabo en forma de videojuego\cite{labster}. ``Al integrar simulaciones en laboratorios virtuales permitimos que los alumnos tengan acceso ilimitado y económico y que practiquen todo lo que quieran hasta que aprendan los conceptos perfectamente'', dice su fundador, el danés Michael Bodekaer.	

\subsubsection{Enfoque del trabajo}
Es entonces nuestra misión llevar esto último al terreno de las telecomunicaciones. A sabiendas de que los videojuegos suponen un portal educativo de gran riqueza, nos proponemos desarrollar un ejemplo de \textbf{juego que permita aprender telemática}.

La telemática ``describe los procesos de transmisión y gestión de informaciones digitales así como los servicios y aplicaciones que se apoyan en ellos''\cite{telematica}. Entre esos servicios se encuentran las redes. La infraestructura asociada a estas suele ser cara y costosa, requiriendo un desembolso importante así como un despliegue inicial no apto para cualquiera. Esto recuerda directamente a las palabras anteriormente expuestas por el fundador de Labster. Su solución fue virtualizar laboratorios e insertarlos en juegos para, así, acercar estos a los estudiantes. Nosotros aquí pretendemos algo similar.

El fin que se propone este trabajo es el de crear un videojuego que integre una red virtualizada. Por suerte para nosotros, a diferencia de Bodekaer, no tenemos que encargarnos de crear un sistema de virtualización de redes, pues, como veremos con detalle en el capítulo \ref{chap:ArtState}, ya existen varios. El objetivo principal pasa por enlazar tal simulador de redes (una herramienta de virtualización de las mismas) con un videojuego.

Precisamente aquel es el primordial problema con el que se habrá de lidiar. A día de hoy, no existe forma directa de unir un videojuego con un simulador de redes. Hay que considerar que estos simuladores están concebidos como finalidad en sí mismos (como una aplicación cualquiera) y no como herramienta para ser usada por utilidades externas. Se requiere por consiguiente desarrollar una infraestructura que haga de nexo entre la red virtual y un videojuego.
 
\subsection{Objetivos}
Aclarado lo anterior, los objetivos que el proyecto se propone son los siguientes:
\begin{itemize}
\item Posibilitar la \textbf{interacción directa entre un videojuego y un simulador de redes}. Aquí reside nuestro papel fundamental del proceso. Con tal propósito crearemos una librería desde la que poder realizar la interacción.
\item \textbf{Desarrollar un videojuego} que haga uso de tal librería; que sea capaz, asimismo, de encontrar un uso didáctico para ella y aplicarlo.
\end{itemize}

\section{Estructura del trabajo}
Para finalizar el capítulo, se expondrá a modo de adelanto una lista que recoge cada capítulo a tratar en el proyecto. Junto a todos ellos, se incluye un pequeño resumen que pretende dar a conocer qué podrá estudiarse en sus respectivas páginas:
\begin{itemize}
\item \textbf{Capítulo \ref{chap:Intro}: Introducción}. En el actual capítulo, como ya se ha podido ver, se han abordado los distintas causas que han motivado a existir a este proyecto a la vez que la forma en la que se intentará llevar a buen puerto su resolución.
\item \textbf{Capítulo \ref{chap:Plan}: Planificación}. Una guía del trabajo llevado a cabo, concerniente tanto a la manera en que se ha distribuido temporalmente como a los utensilios empleados.
\item \textbf{Capítulo \ref{chap:ArtState}: Estado del arte}. Las distintas tecnologías barajadas sobre las que trabajar son aquí listadas.
\item \textbf{Capítulo \ref{chap:Analisis}: Análisis tecnológico}. Tras citar algunos ejemplos de tecnologías que podrían emplearse para el proyecto en el anterior capítulos, aquí se hace un análisis más en profundidad de las que han resultado escogidas.
\item \textbf{Capítulo \ref{chap:Design}: Diseño}. Se determina la forma en la que se llevará a cabo la solución del problema inicial.
\item \textbf{Capítulo \ref{chap:Integration}: Implementación}. Trazado un camino a seguir en el capítulo anterior, en este se describe la manera en la que este camino toma forma. Se muestra qué se ha creado finalmente y cómo se ha logrado.
\item \textbf{Capítulo \ref{chap:Pruebas}: Pruebas}. En este capítulo el resultado final es mostrado y se añade una valoración del mismo.
\item \textbf{Capítulo \ref{chap:Conclusiones}: Conclusiones}. Para finalizar el proyecto, se ha hecho un resumen del trabajo realizado junto a una valoración del mismo. Además, algunas propuestas de trabajos futuros sobre él son dadas.
\end{itemize}

\chapter{Planificación}\label{chap:Plan}

fgxdh

\section{Planificación temporal}

huigk

\section{Recursos empleados}
Dividiremos los recursos utilizados para la realización del trabajo en tres: recursos humanos, recursos software y recursos hardware:

\subsection{Recursos humanos}
Aquí una lista de las personas que han participado en el proyecto y su papel en el mismo.
\begin{itemize}
\item Juan José Ramos Muñoz, profesor del departamento de Teoría de
la Señal, Telemática y Comunicaciones en la ETSIIT de la Universidad de Granada. Tutor del proyecto
\item Jonathan Prados Garzón, profesor del departamento de Teoría de
la Señal, Telemática y Comunicaciones en la ETSIIT de la Universidad de Granada.
Cotutor del proyecto.
\item Ángel Oreste Rodríguez Romero, alumno del grado de ingeniería de tecnologías de telecomunicación en la ETSIIT de la Universidad de Granada. Autor del proyecto.
\end{itemize}

\subsection{Recursos software}
Aplicaciones y demás material digital que ha sido necesario para desarrollar el trabajo:
\begin{itemize}
\item \href{https://www.microsoft.com/es-es/windows}{\textbf{Microsoft Windows 10 Home versión 1803}} como sistema operativo sobre el que se ha llevado todo el proyecto a cabo.
\item \href{https://www.mozilla.org/es-ES/firefox/}{\textbf{Mozilla Firefox Quantum}} como navegador web. 61.0.2 es la última versión empleada.
\item Para la realización de la memoria:
\begin{itemize}
\item El editor \href{http://www.xm1math.net/texmaker/}{\textbf{Texmaker 5.0.2}} junto a \href{https://miktex.org/about}{MiKTeX 2.9}, implementación de \LaTeX.
\item \href{https://es.libreoffice.org/descubre/draw/}{\textbf{LibreOffice Draw 6.1.0.3}} para el diseño de algunos esquemas.
\item \href{https://inkscape.org/es/}{\textbf{InkScape 0.92}} para la transformación de imágenes vectoriales en archivos de formato \textit{.pdf}.
\end{itemize}
\item Como lector de archivos \textit{.pdf} ha sido elegido \href{https://www.foxitsoftware.com/pdf-reader/}{\textbf{Foxit Reader}}. Última versión: 9.2.0.
\item Para el desarrollo de código:
\begin{itemize}
\item El editor más utilizado durante el transcurso del trabajo ha sido \href{https://code.visualstudio.com/}{\textbf{Visual Studio Code}}. Ha pasado por varias versiones durante su uso. La más reciente es 1.26.
\item Para la compilación de código y el scripting en Unity el hermano mayor del anterior, \href{https://visualstudio.microsoft.com/es/}{\textbf{Microsoft Visual Studio}}, también ha pasado por varias versiones, pero la última de ellas ha sido la 15.8.0.
\item \href{https://notepad-plus-plus.org/}{\textbf{Notepad++}} ha sido utilizado para tomar notas y revisar código de forma rápida y ligera. La última versión usada es la 7.5.8.
\item \href{https://www.visual-paradigm.com/}{Visual Paradigm Enterprise 15.1} para dibujar los diagramas UML.
\end{itemize}
\item Para la virtualización de redes:
\begin{itemize}
\item \href{https://www.gns3.com/}{\textbf{GNS3 2.1.3}} como emulador de redes.
\item \href{https://www.virtualbox.org/}{\textbf{VirtualBox 5.2.16}} para la virtualización de ciertos dispositivos de la red.
\end{itemize}
\item El motor para el desarrollo de videojuegos \href{https://unity3d.com/es}{\textbf{Unity}} versión personal. El último número de versión empleado ha sido el 2018.2.2f1.
\end{itemize}

\subsection{Recursos hardware}
Todo el proyecto ha sido construido a través del portátil Lenovo Z500. Cuenta con un procesador Intel i7 3232QM de ocho núcleos virtuales, 16GB de RAM DDR3 a 798MHz, un disco duro HDD de 1GB a 5400RPM y una tarjeta gráfica dedicada Nvidia 740M con 1GB VRAM.

\section{Desembolso económico}




\chapter{Estado del arte}\label{chap:ArtState}

El estado del arte se define como el nivel de desarrollo de un ámbito concreto, generalmente relacionado con el mundo técnico-científico.

\section{Simuladores de redes}

\subsection{Simulador 1}

\subsection{Simulador 2}

\subsection{Simulador 3}

\subsection{GNS3}


\section{Motores de videojuegos}
Pondría una subsección por cada motor a considerar. Y en cada uno, comentar por qué es adecuado o no: precio, licencia, curva de aprendizaje, plataformas que soporta, facilidad para añadir librerías, o lenguaje que utilice que sea más o menos apropiado, si hay mecanismos para comunicarte con software externo (como el caso del emulador)... Con media página por cada uno, creoque basta. Unreal Engine, Amazon Lumberyard, Game Maker, Godot Engine, y no sé si alguno más que sea famosete, indicando que son los más populares (ojalá haya un informe con lista de usuarios por engine...).

Creo que tanto para hablar de Unity como de GNS3, lo suyo es:

comentar las características generales en "motores de juegos" y en la de "emuladores". y luego añadir una sección quizás con los detalles de las plataformas elegidas, hablando de las características que vas a utilizar (API de GNS3, capacidad de enlazado con DLL de Unity...). Cosas que vaya a necesitar el revisor para entender el diseño o sobre todo la integración.

\subsection{Unreal Engine}

\subsection{Amazon Lumberyard}

\subsection{Game Maker}

\subsection{Godot Engine}

\subsection{Unity}

\section{Juegos docentes}
Deberían incluirse juegos que ya existan. Añade una subsección "conclusiones", o algo en la que compares qué tienen y debe incorporar tu juego, y qué cosas les falta, que vas a poner en el tuyo.
\chapter{Analisis}\label{chap:Analisis}
En el capítulo anterior se listaron distintas tecnologías que podían sernos útiles para nuestro proyecto. En el actual nos detendremos a explicar cuáles de ellas explicaremos y el por qué de su elección.

\section{Simulador de redes: GNS3}
Se ha decidido que GNS3 sea el simulador de redes a utilizar para nuestro proyecto por varias razones:
\begin{itemize}
\item Es \textbf{multiplataforma}, con lo que podemos trabajar sencillamente con él tanto desde Windows como Linux.
\item Existe \textbf{mucha información} para consultar. Desde internet contamos con documentación oficial y un foro propio. Hay incluso libros sobre él y como ejemplo tenemos el usado en varias ocasiones en este documento, ``The Book of GNS3'', por Jason C. Neumann.
\item Es altamente \textbf{expandible}: hay decenas de aparatos reales que pueden incluirse y emularse en las topologías creadas. Pueden descargarse desde \MYhref{https://www.gns3.com/marketplace/appliances}{su marketplace}.
\item \textbf{Cuenta con una API REST} que lo hace interactivo desde el exterior.
\end{itemize}

GNS3 se estructura en \textbf{proyectos}. En cada proyecto podemos desplegar una serie de dispositivos y conectarlos entre sí. Al guardar un proyecto como tal, el despliegue realizado se guarda junto a él. Sin embargo, el estado de las máquinas no corre la misma suerte, y es que a cada apagado de las mismas toda su configuración se pierde.

Una de las características más interesantes de GNS3 y sus proyectos está en la importación y exportación de estos. El simulador cuenta con una opción que permite extraer todo un proyecto en una imagen (de formato \textit{.gns3project} para que sea fácilmente portable. Existe la posibilidad de exportar, junto a la topología creada, las imágenes base sobre las que los nodos del proyecto están construidas para que no sea necesario contar con esas imágenes en GNS3 donde se pretende importar.

\begin{figure}[h]
  \centering
  \includegraphics[scale=0.15]{imagenes/interfazgns}
  \caption{Interfaz de GNS3}
  \label{fig:interfazgns}
\end{figure}

La interfaz de usuario de la aplicación puede verse en la figura \ref{fig:interfazgns}. La barra de arriba está repleta de opciones relacionadas con el proyecto, como arrancar todos los dispositivos, pausarlos, pararlos o incluso herramientas de dibujo para convertir el esquema en algo más intuitivo y legible. A la izquierda se listan los nodos disponibles en la máquina. Arrastrándolos al centro los insertamos en el proyecto.

A continuación pasaremos a explicar conceptos clave del emulador.

\subsection{El servidor}

\subsubsection{La API}
Una API (que será explicada más adelante)
La primera vez que aparezca un acrónimo, debes indicar cuál es su significado. De hecho, en los títulos o como primera palabra de la frase, (o en el abstract) hay que evitar las abreviaturas.

\subsubsection{Nodos}
Cada elemento de una red está representado en GNS3 por un elemento llamado \textbf{nodo}. Estos nodos, que pueden ser desde un router a un switch, no son más que virtualizaciones de aparatos reales. Por norma general, estas virtualizaciones se realizan a partir de imágenes de los sistemas operativos que se integran en los aparatos. Así, podemos tener varios routers distintos de Cisco montados sobre la misma estructura, permitiéndonos jugar con ellos con verdadera facilidad.

GNS3 es usado ampliamente como método de entrenamiento para los exámenes de Cisco. Tal es así, que en su academia se pueden encontrar cursos para facilitar la \MYhref{https://academy.gns3.com/p/the-complete-networking-fundamentals-course-your-ccna-start/?product_id=169831&coupon_code=HOMEPAGE}{obtención del CCNA}. Sin embargo, las imágenes de las máquinas de Cisco, aunque pueden ser encontradas fácilmente en internet, requieren de licencia para ser usadas. Siendo así se optó por hacer uso de software libre para el proyecto.

\subsubsection{Enlaces}

\section{Motor de videojuegos: Unity}
realmente esta elección no era tan complicada
\chapter{Diseño de la solución}\label{chap:Design}
En este capítulo se describirá el modo en el que la API será construida y cómo el juego ha de ser llevado a cabo para cumplir con los propósitos que se han ido repitiendo a lo largo del texto.

\section{La API: GNS3sharp}\label{sec:dis_api}
Para llevar a cabo la interacción entre el videojuego y el simulador de redes es necesario desarrollar previamente alguna herramienta que lo posibilite. Con esta fin, se ha construido una librería propia que lo posibilita.

\begin{figure}[H]
  \centering
  \includegraphics[scale=1.4]{imagenes/capa_api}
  \caption{La API en el proyecto}
  \label{fig:capa_api}
\end{figure}

El lugar de esta API (aquí usaremos indistintamente API y librería) en nuestro proyecto se encuentra entre el juego y GNS3 tal y como se deja entrever en la figura \ref{fig:capa_api}. Todos sus detalles serán explicados en las próximas páginas.

\subsection{Introducción}
Según Wikipedia, una \textbf{API} ``es un conjunto de subrutinas, funciones y procedimientos (o métodos, en la programación orientada a objetos) que ofrece cierta biblioteca para ser utilizado por otro software como una capa de abstracción''\cite{wiki:api}. Comprende una serie de funciones que facilitan en mayor o menor medida trabajar sobre un cierto elemento. Como ejemplo de API famosa tenemos la de Google Maps, que contiene un compendio de métodos para JavaScript que permiten interactuar directamente con la plataforma de Google y crear nuestros programas jugando con ella\cite{googlemaps}. En resumen, una API no es más que un conjunto de funciones definidas para acceder a un servicio determinado.

La programación se rige por \textbf{capas de abstracción}. Partiendo de conceptos concretos se desarrolla una capa de abstracción haciendo uso de ellos que permite elevar el trato de elementos concretos un nivel por encima\cite{abstraction}. De esta forma ganamos en agilidad de escritura y sencillez en el diseño sin perder flexibilidad de desarrollo.

\textit{Pondría más bien que para facilitar esta comunicación, una de las alternativas que exsiten en sistemas complejos, es aplicar el concepto de capa. Una capa sería.. (en redes, el modelo OSI o el TCP/IP usan capas, en la que la superior solicita servicios a la inferior, y esta se lo facilita... En TCP/IP se define de forma abstracta, no a nivel de programación).}

Como se ha citado previamente, GNS3 hace uso de una \textbf{API REST} (\textit{REpresentational State Transfer}). Esto es, mediante una serie de métodos (los conocidos GET, POST...) asociados a una URI concreta podemos interactuar \textbf{vía web} con una aplicación. La diferencia fundamental con otra clase de servicios web es que REST está orientado a recursos y no a métodos. Esto permite a la web utilizar comunicaciones sin estado, facilitando de este modo su escalabilidad\cite{REST}.

Aunque de gran utilidad (ya veremos qué papel concreto juega en nuestro trabajo), hace falta algo más para poder propiciar la interacción entre el simulador de redes y el videojuego; la API trabaja a un nivel mucho más bajo del que requerimos para desarrollar los juegos. Surge la necesidad entonces de crear nuestra propia API que haga uso de la API REST de GNS3 y que defina las funciones necesarias, a un nivel más abstracto y con más facilidades, que haga de intermediaria entre GNS3 y Unity.

\subsection{Elección del lenguaje}
En pleno 2018 la cantidad de lenguajes de programación existentes roza el absurdo. Desde el tradicional C, pasando por el multifuncional Java, el sencillo Python o incluso los llamados lenguajes esotéricos como LOLCODE\cite{esotericlang}. De entre todos ellos nosotros elegiremos uno sobre el que trabajar. Esta decisión está condicionada, como es natural, por el motor de videojuegos a utilizar.

Ya que nuestra intención es que el motor sea capaz de establecer interacción con el simulador, es necesario que la API a desarrollar esté escrita en un lenguaje con el que el propio motor sea capaz de trabajar. \textbf{C\#} parecía la opción más sensata. ¿Por qué? Las razones se exponen a continuación:

\begin{itemize}
\item\textbf{Porque es el lenguaje más usado en Unity}. Unity admite varios lenguajes de programación con los que desarrollar los scripts asociados a los juegos. C++, usado en otros muchísimos otros motores de videojuegos como Unreal Engine, es uno de ellos. Aunque se trate de un lenguaje increíblemente potente y eficiente, su complejidad de uso es mucho mayor, ya que se encuentra a más bajo nivel. JavaScript es otro de ellos, pero no es tan recomendable como C\#, pues entre otras razones, a diferencia de JS, C\# es fuertemente tipado\cite{unityinaction}. En general, C\# es el lenguaje usado por defecto en Unity y el recomendado por todos, documentación incluida.
\item\textbf{Porque son más motores quienes lo utilizan}. Unity está en el podio de entre los motores de videojuegos más usados en el mundo. Como tal se convierte en un referente. El resto de motores miran hacia él y, si quieren atraer a nuevos programadores, tendrán que hacer de su incursión en el nuevo motor algo sencillo. Este es el caso de Godot Engine, que unos meses atrás decidió incluir tal lenguaje entre los soportados\cite{godotcs}. Esto quiere decir que la API no solo podrá ser usada en juegos creados en Unity, si no que su terreno de juego se verá ampliado. CryEngine es otro de los motores que permiten el uso de C\# como lenguaje de scripting.
\item\textbf{Porque, ante todo, es un gran lenguaje}. C\#, similar en cuanto a sintaxis a Java, nació como respuesta a este de la mano de Microsoft. Se trata de un lenguaje de propósito general, aunque es usado primordialmente para la construcción de aplicaciones para infraestructuras Windows. Es uno de los lenguajes que componen la plataforma .NET de Microsoft. Tal es su importancia que a día de hoy se posiciona como el cuarto lenguaje de programación más usado a nivel mundial\cite{csisfamous}. Alguna de las características que lo hacen especialmente atractivo:
\begin{itemize}
\item Es un \textbf{lenguaje de programación orientado a objetos}, con lo que posee todas las características propias de estos (encapsulado, herencia y poliformismo). Sin embargo, no admite multiherencia. Su componente fundamental es una unidad de encapsulamiento de datos y
funciones llamada \texttt{type} o ``tipo''. C\# tiene un sistema de tipos unificado, donde todos los tipos en última instancia comparten un tipo de base común.
\item Aunque es principalmente un lenguaje orientado a objetos, también \textbf{toma prestado del paradigma de programación funcional}. Las funciones pueden ser tratadas como valores mediante el uso de delegados, permite el uso de expresiones lambda, acercándose a los patrones declarativos del paradigma funcional...
\item Admite \textbf{tipado estatico}, lo cual se traduce en que el lenguaje obliga a que haya coherencia entre los tipos durante el tiempo de compilado. El tipado estático elimina un gran número de errores antes de que se ejecute un programa. Desplaza la carga del momento de la ejecución hacia el compilador para verificar que todos los tipos en un programa encajan correctamente. Esto hace que las aplicaciones grandes sean mucho más fáciles de administrar, más predecibles y más robustas. Además, la escritura estática permite que herramientas como IntelliSense en Visual Studio ayuden a desarrollar, pues conoce el tipo de una variable determinada y, por lo tanto, qué métodos puede utilizar está habilitada a usar. C\# incluye además el tipo \texttt{dynamic} que permite sortear el tipado estático y dejar que el tipo de variable se averigüe durante el momento de la ejecución\cite{csnutshell}.
\end{itemize}
Particularmente, nosotros haremos uso de C\#7, lanzado a la vez que Visual Studio 2017. Entre sus características más interesantes se incluye la aparición de tuplas, especialmente útil cuando queremos devolver más de una variable como resultado de una función.
\end{itemize}
Aclarada las razones, pasamos a analizar el uso que le daremos a este lenguaje.

\subsection{Diseño de la API}
Antes de desarrollar la API, que se ha decidido llamar \textbf{GNS3sharp} por la fusión entre las dos tecnologías que la conforman, es necesario reflexionar sobre la forma que tendrá. Para dar soluciones, tenemos que plantearnos antes las preguntas adecuadas:
\subsubsection[Interacción con GNS3]{¿Cómo debe interactuar con el simulador?}
Como ya se dijo con anterioridad, GNS3 crea un servidor en el equipo desde el que se ejecuta. Este servidor acepta peticiones REST, creando así una suerte de interacción con el programa sin necesidad de poner las manos en él de forma directa. Se abre de esta forma al mundo del scripting y así a nosotros para crear un conjunto de métodos que faciliten su acceso y gestión. Todo esto volverá a ser ilustrado en la subsección \ref{subsec:interac_emul} con mayor detalle.

Tratándose de una API REST, lo único que nos es necesario para la conexión es un \textbf{cliente web}. El framework de .NET ya cuenta con una clase dedicada a ello denominada \texttt{System.Net.WebClient}\cite{webclient}, que contiene métodos más que suficientes para nuestro uso. Será pues sobre esto sobre lo que se base primordialmente la relación entre nuestra librería y el simulador de redes como tal. A continuación veremos que habremos de emplear alguna tecnología más.

\subsubsection[Acceso a la API]{¿Cómo queremos que interactúe con el exterior?}\label{subsec:accesoapi}
Es importante tener esto claro, pues el uso de la librería y de sus métodos por parte de una aplicación externa debe optimizar la productividad y agilizar el desarrollo lo máximo posible. Aquel que vaya a hacer uso de ellas debe tener un esquema claro del modo en que puede gestionar la interacción, siendo pues cercana al usuario (más bien desarrollador) último.

GNS3 funciona a través de proyectos. En un proyecto se pueden incluir tantos nodos como se desee para, a posteriori, ser interconectados. Las posibilidades son muy amplias y dejan poder absoluto al usuario. La singularidad del proyecto apenas pasa por permitir arrancar o apagar todos los nodos contenidos en él a la vez.

Siendo así, se ha optado por concentrar lo crucial de la interacción entre el simulador y el desarrollador en \textbf{una sola clase-objeto}. Esta clase, que se explicará con detalle más adelante, permitirá hacer de puente entre el proyecto y los distintos elementos que lo componen y aquel que haga uso de ella.

\subsubsection[Funcionalidad]{Más importante aún, ¿qué queremos que haga?}
Todo lo explicado hasta ahora es correcto: definimos una forma con la que el código pueda acceder al simulador y reflexionamos en la manera en la que el desarrollador que la use pueda sacarle partido. Sin embargo, ¿qué significa entonces \textit{sacarle partido}?

Para responder no hay más que irse al propio GNS3 y ver todas las opciones que nos ofrece: desplegar nodos, enlazarlos entre sí, cortar esos enlaces, arrancar todos los nodos juntos, parar algunos de ellos a voluntad... 

Y por supuesto y aún más importante, \textbf{gestionar los aparatos desplegados desde dentro}. Es aquí donde está su verdadero potencial. Manejar un switch conectado a varios PCs, modificar sus VLANs dinámicamente; un router conectado a varias redes distintas, desactivando y activando algunas de ellas en función de un booleano con el que estemos trabajando... Y todo esto sin necesidad de acceder a GNS3 directamente; todo mediante scripting. ¡Se abre un mundo de posibilidades!

\subsection{Diseño de clases}
Nuestra API hará uso de todas las características propias de los lenguajes orientados a objetos ya expuestas. A nivel de estructura, la que nos interesa citar ahora es la \textbf{herencia}.

La herencia es un mecanismo por el que una clase hija (llamémosla \texttt{B}) va a heredar los métodos y las propiedades de una clase padre (llamémosla \texttt{A}). La cantidad de elementos heredados entre padre e hijo puede ser determinado mediante el uso de modificadores de acceso. Como dato adicional, este lenguaje no admite herencia de constructores, así que nuestra clase \texttt{B} tendrá que definir su propio constructor (o bien explicitar su herencia respecto al de \texttt{A}).

Aclarado el concepto, mostramos la estructura básica sobre la que nuestra librería será construida:

\begin{figure}[H]
  \centering
  \includegraphics[scale=0.75]{imagenes/diagrama_api1}
  \caption{Boceto de diagrama UML de la API}
  \label{fig:diagrama_api1}
\end{figure}

El diagrama UML de la figura \ref{fig:diagrama_api1} dibuja cuáles son las clases que componen la librería y da pistas acerca de la relación entre ellas. Así, es claro que las tales \texttt{Switch}, \texttt{Node} y \texttt{Router} heredan de de \NODE~y que a su vez esta es usada por \GNSCS~por alguna propiedad llamada \texttt{Nodes} (que hasta la subsección \ref{subsec:gnscsclass} no se analizará).

A continuación pasamos a desarrollar cada una de las clases conceptualmente:

\subsubsection[''La clase principal'']{GNS3sharp, la clase principal}
Tal y como expusimos en la sección~\ref{subsec:accesoapi}, crearemos una clase principal que gestionará toda la interacción con el simulador de redes. Hemos llamado a esta clase del mismo modo que a la API, \GNSCS. Recalcamos entonces: los objetos que se encarguen de gestionar los proyectos serán del tipo \GNSCS.

¿Y en qué consistirá esta gestión concretamente? Básicamente, la clase debe encargarse de:
\begin{enumerate}
\item Establecer la conexión con el servidor de GNS3 y recopilar toda la información acerca del proyecto que se pretende controlar.
\item Convertir toda es información en objetos útiles que puedan ser utilizados.
\item Crear una gestión eficiente de esos recursos de manera que sean fácilmente accesibles y manipulables.
\end{enumerate}

Estos recursos de los que hablamos serán en gran medida los representados por las clases que se desarrollarán justo debajo.

Sin embargo, la creación de proyectos y despliegue de nodos en los mismos quedarán excluidos, ya sea por la complejidad añadida que conlleva o porque no son funciones que nos sean vitales para la construcción de juegos. Estas funcionalidades deberán ser llevadas a cabo manualmente. Más adelante podrán ser controladas desde la librería sin problema.

\subsubsection{Node}
Sin lugar a dudas es aquí donde se encuentra la característica más interesante de GNS3 y es hacia donde nuestros esfuerzos deben dirigirse. Dado que cada nodo representa un aparato de la red, lo ideal es que ese aparato pueda ser convertido a un objeto en nuestra biblioteca desde el que se nos permita su control. 

Esta es la finalidad de la clase \texttt{Nodo}. Su deber será el de contener todos los parámetros necesarios para habilitar la conexión con el nodo y así abrir un canal de comunicación con él; un canal que habilite tanto el envío como la recepción de mensajes.

Para la creación de las instancias de la clase se tendrá que recurrir a \GNSCS, que mediante los datos que recoja del servidor de GNS3 será capaz de crear asimismo el objeto.

Como GNS3 admite todo tipo de aparatos de red, cada uno con sus peculiaridades, lo más sensato es crear una clase para cada uno de esos equipos. Esta individualización permite definir métodos propios para cada elemento y, de este modo, facilitar su uso final. Tal y como se puede ver en el diagrama de la figura~\ref{fig:diagrama_api1}, definiremos tres clases principales:
\begin{enumerate}
\item \texttt{Guest}, representante de dispositivos \textit{end-point} como son los PCs.
\item \texttt{Router}, representante de dispositivos de la tercera capa OSI.
\item \texttt{Switch}, representantes de dispositivos de la segunda capa OSI.
\end{enumerate}

Todas ellas heredarán de \NODE. De estas clases nacerán asimismo otra serie de clases referidas a aparatos concretos y a no tipos genéricos.

\subsubsection{Link}
Los nodos están interconectados entre sí mediante enlaces. Representaremos cada uno de ellos mediante la clase \LINK. Esta clase guardará como objetos a los nodos que interconecta así como información sobre el propio enlace (cuánto jitter, latencia... posee). Al igual que \LINK, la generación de instancias de esta clase pasará por \GNSCS.

La clave de representar los enlaces en la API es que nos permitirá añadir, eliminar o incluso eliminarlos del proyecto desde el código, ampliando las posibilidades de interacción con el simulador.

\subsubsection{Otras clases}
Además de las anteriormente expuestas, se creará otra serie de clases que, o bien ayuden al desarrollo de la API (una clase con funciones de ayuda), o bien faciliten la representación de los datos (tal será el caso de \texttt{RoutingTable}, que puede atisbarse en la figura~\ref{fig:diagrama_api1}).

\section{Diseño del videojuego}
Establecida una plataforma sobre la que GNS3 pueda interactuar con código, llega el momento de diseñar el videojuego que haga uso de ella. 

\subsection{Interacción con el simulador}\label{subsec:interac_emul}
A continuación se dará una visión de la interacción desde el punto de vista del motor del videojuego; desde la creación del juego. Se facilita el siguiente esquema a la espera de que sea de ayuda a la hora de comprender la sección.

\begin{figure}[H]
  \centering
  \includegraphics[scale=1.4]{imagenes/diagrama_interaccion}
  \caption{Diagrama de la interacción entre Unity y GNS3}
  \label{fig:diagrama_interaccion}
\end{figure}

Los proyectos de Unity están basados en un elemento llamado \GAOBJ. Todos los objetos que componen las escenas del proyecto se pueden abstraer en última instancia en ese tipo. El primer paso para que un proyecto de Unity pueda conectarse y controlar un esquema de red de GNS3 pasa por \textbf{insertar un \GAOBJ~ que tenga una instancia de la clase \GNSCS}. La única condición que se le impone a este objeto es que este parámetro contenido en él (la instancia) ha de ser accesible por todo el resto de elementos de la escena.

¿Para qué? De este modo, todos esos elementos no necesitan crear la suya propia. Creando una sola instancia de la clase centralizamos el trabajo en un solo objeto. La instancia contendrá información de los datos del proyecto de GNS3; información que podrá ser tomada por todos los componentes de la escena sin más que copiar la referencia al objeto. Si estos declararán su propia instancia cada uno perdería en tiempo (la recopilación de datos desde el servidor no es instantánea) y en eficiencia.

La idea es que el jugador (en otros términos, el usuario final), posea interacción con el simulador de redes de forma que \textbf{las consecuencias puedan verse reflejadas en el juego}. Pongamos un ejemplo e invirtamos el orden lógico de desarrollo para entender mejor cómo funcionaría esto:

\begin{enumerate}
\item El jugador se encuentra en una habitación que no es más que una \textbf{representación de un router}. Tiene un botón justo delante suya. En la pantalla se le avisa de que si lo pulsa, la puerta asociada a una determinada interfaz se cerrará, de forma que los enemigos que se ven acercarse en esa dirección no podrán pasar. Lo pulsa entonces.
\item Ese botón es un \GAOBJ~ de la escena actual que tiene un \textit{trigger} asociado a él y un script que lo gestiona. Uno de los atributos de tal código es la referencia a la instancia del objeto \GNSCS, construido al arrancar la escena y que suponemos ya contiene toda la información del proyecto de GNS3. Cuando se activa el trigger (una condición del script pasa a ser \texttt{True}), se toma la referencia del objeto asociado al router donde el personaje está. Se utiliza tal objeto para enviar un comando al router que le haga apagar la interfaz correspondiente.
\item El mensaje llega al router. Este lo ejecuta y devuelve el resultado.
\item El script recibe la respuesta y, tras analizarla, confirma que la interfaz se ha cerrado. Envía un mensaje al \GAOBJ~ de la puerta para que la cierre.
\end{enumerate}

Este flujo es de alguna forma lo que se ha representado en la figura~\ref{fig:diagrama_interaccion}. El bloque morado hace referencia a todo el entorno que comprende una escena de Unity. Esta escena contendrá una serie de \GAOBJ s, cuyo comportamiento vendrá determinado mayormente mediante lógica programada por scripts. El bloque azul hace referencia a este compendio. Alguno de esos scripts será el encargado de contener una instancia de \GNSCS . Al ser creada, esta se conecta al servidor de GNS3 desde donde recibe toda la información del proyecto con el que se quiere trabajar. A partir de ahí, el resto de \GAOBJ s de la escena no tendrán más que usar esa instancia para poder conectarse a los nodos (entre otras cosas) de la simulación. 

\subsection{Modelo de videojuego}\label{subsec:modelojuego}
Ahora que sabemos de qué forma se interconecta el juego con la red y en qué consiste exactamente la interacción, queda por concretizar qué clase de juegos podríamos crear. Hay que tener en mente en todo momento que pretendemos crear un \textbf{juego didáctico}, así que no nos vale cualquier construcción. La dificultad es, entonces, doble, pues es necesario seguir los cánones que dictan el buen hacer de un videojuego (véase un inteligente diseño de niveles, una estética atractiva y coherente, apartado técnico gráfico y sonoro propio, niveles a prueba de bugs...) a la vez que planear \textbf{qué} se quiere enseñar en ellos y \textbf{cómo} se pretende hacerlo.

Para sortear estas dificultades se ha decidido que la estructura del juego esté basada en pequeños niveles interconectados entre sí por una plataforma común. La linealidad de un juego exige de un guion y de una cohesividad de los que el juego ``federado'' puede permitirse  prescindir.

\begin{figure}[H]
  \centering
  \includegraphics[scale=0.35, trim={0 9cm 0 9cm}]{imagenes/hall_juego}
  \caption{Boceto del juego}
  \label{fig:hall}
\end{figure}

En el boceto de la figura de arriba se quiere ejemplificar (de forma extremadamente simplista) en qué consistiría este diseño.
\begin{itemize}
\item Cada vez que el jugador entre en el juego se encontrará en una suerte de ``hall''. El hall estará acotado por paredes con un número determinado de puertas.
\item Estas puertas conducen a los distintos niveles del juego. Cada puerta corresponde a una lección diferente sobre redes y telemática. El jugador no tiene que superar una lección (un nivel, una zona) para acceder a la siguiente, sino que puede elegir qué quiere aprender indistintamente. Por supuesto, se podría vetar la entrada a una zona hasta que supere una concreta. Idealmente, cada nivel al que se accede contaría con varias fases que van incrementando en dificultad.
\item Al superar estas zonas, se desbloquea un ``manual'' acerca de lo aprendido en esa sala. Este manual puede ser consultado en un monitor localizado en el centro del hall. Gracias a esto, se podría repasar rápidamente lo aprendido en niveles anteriores desde un solo lugar.
\end{itemize}
Esta forma de acceder a los niveles es similar a la que siguen los videojuegos de la saga \textit{Crash Bandicoot} (\textit{2}, \textit{3}, \textit{La venganza de Cortex...}), donde existe una sala de ``descanso'' desde la que elegir cuál será la siguiente fase a jugar. Abstrayendo esta idea algo más, no es más que un menú estático desde donde elegir el nivel (juegos de corte más ``clásico'' como \textit{Super Meat Boy} utilizan este patrón) salvo que gozando de una interacción más directa y natural con el jugador, integrándolo más.

Existen dos aproximaciones al modo en el que estos niveles se conectan con el simulador de redes:
\begin{itemize}
\item Por un lado, podría existir \textbf{un proyecto de GNS3 por cada nivel} y/o fase. Aunque mucho más estructurado y por ende posiblemente más sencillo de gestionar, es necesario establecer conexión con el servidor de GNS3 por cada nivel para recopilar la información del mismo. Esto se traduce en tiempo. Además, el servidor de GNS3 no registra el proyecto de GNS3 a menos que este se abra manualmente desde GNS3 en una sesión. Más tiempo.
\item La otra solución pasa por crear \textbf{un solo proyecto} para todo el juego. Sería necesario desplegar todos los aparatos que vayan a ser utilizados en él, incrementando altamente los recursos que el sistema utiliza. Para paliar este último problema existen dos opciones:
\begin{enumerate}
\item Encender los aparatos que vayan a ser necesitados en el nivel a jugar y cerciorarse de que el resto permanece apagado.
\item Reconfigurar los aparatos que vayan a ser utilizados para que sus propiedades se correspondan a lo que se necesita para el respectivo nivel.
\end{enumerate}
De entre estos dos acercamientos al problema, quizás el primero sea más simple, pero el arranque de cada nodo no es inmmediato. El segundo entraña una mayor dificultad ingenieril y mayor propensión al error aunque la configuración de los aparatos conlleve menos tiempo.
\end{itemize}

Sin embargo, aunque todas estas ideas pueden sonar realmente interesantes, el juego final desarrollado no ha podido alcanzarlas. En el siguiente capítulo, donde se detalla la integración real de lo expuesto en este capítulo, se comprobarán cuáles han sido las dificultas encontradas en el camino que han frenado de una forma u otra el escenario inicial. 
\input{capitulos/06_Implementacion}
\chapter{Pruebas}\label{chap:Pruebas}
En el penúltimo capítulo del documento se mostrarán los resultados del proyecto; se hará un análisis del establecimiento del juego y del rendimiento del simulador que lo soporta.

\section{Descripción de las pruebas}
Una vez asentado todo lo que vimos en el capítulo \ref{chap:Integration}, vamos a probar su funcionamiento así como su rendimiento. Hemos decidido llevar a cabo las pruebas en dos entornos diferentes, para ponderar ventajas e inconvenientes de cada uno de ellos. Estos se describen a continuación:
\begin{itemize}
\item El \textbf{caso 1} es sobre el que todo el desarrollo del proyecto ha sido ejecutado. El entorno consiste en una máquina física con Windows 10 desde donde el juego es ejecutado y GNS3 debe estar corriendo. Como se trata de un sistema operativo Windows, nodos como los de OpenWRT necesitan ser montados sobre un GNS3 VM, lo cual implica hacer uso de un hipervisor (aquí usaremos Virtualbox, pues aunque su rendimiento es inferior al de VMware, es open source) donde instalarlo. El proceso de despliegue es entonces algo tedioso, ya que requiere de una instalación de GNS3, una configuración apropiada del mismo y una máquina virtual con GNS3 VM.
\item El \textbf{caso 2} busca acabar con ese proceso. La idea aquí es usar una sola máquina virtual como contenedor de todo lo que el juego necesita del simulador. Para ello, se ha preparado una máquina virtual con Xubuntu (recalcar que el adaptador de red está establecido como ``puente'', ya que así que la máquina toma una dirección de red diferente de aquella de la máquina huésped) y sobre él se ha instalado GNS3; se ha configurado este, se ha creado el proyecto de la figura \ref{fig:esquematico_red} y finalmente se ha exportado la máquina en formato \textit{.ova}. De esta forma ganamos en portabilidad. Además, puede demostrar que la API funciona incluso cuando el GNS3 al que llama es remoto.

La máquina virtual tiene asignados dos núcleos del procesador.
\end{itemize}

En ambos casos, se llevarán a cabo varias pruebas sobre el proyecto de la figura \ref{fig:esquematico_red}.
\begin{itemize}
\item La primera de ellas observará el consumo de RAM que GNS3 toma, tanto cuando el proyecto está parado como en acción. Para ello son usadas dos herramientas de Microsoft: Process Explorer y RamMap. Usaríamos únicamente el primero de ellos si no fuera porque los hipervisores utilizan algo llamado ``memoria bloqueada por el controlador'' (\textit{driver locked memory}) que los monitores de recursos normales no muestran. La memoria bloqueada por el controlador aparece cuando un controlador modo-kernel evita que las páginas de memoria sean cambiadas al archivo de paginación. Es a través de este mecanismo que el hipervisor varía la cantidad de memoria disponible para un huésped cuando la memoria dinámica está activada \cite{dlm}.
\item La segunda medirá cuánto tiempo tarda cada router OpenWRT en iniciarse de forma aislada (se arranca él solo, no el proyecto completo). Para ello, se empieza a cronometrar desde que se pulsa el botón de inicio del nodo hasta que en el terminal del router aparece el mensaje \textit{br-lan: port 1(eth0) entered forwarding state}, a partir del cual el nodo comienza a ser plenamente funcional. También se testeará el caso análogo para cuando se ejecuta el proyecto completo. En esta situación, se mostrará el tiempo transcurrido hasta que todos los routers sean correctamente inicializados.
\item La última medirá el tiempo que Unity necesita para cargar la escena tomando los datos del simulador. No se contará aquí el la espera establecida manualmente para que el juego no trabaje hasta que los aparatos estén disponibles. Extraeremos tres datos:
\begin{enumerate}
\item Tiempo de establecimiento de los VPCs: cuánto tiempo se necesita para que su configuración sea establecida en GNS3.
\item Tiempo de establecimiento de los routers: cuánto tiempo se necesita para que su configuración sea establecida en GNS3. Recordar que esta tarea fue paralelizada de modo que se establecían todos ``a la vez'' y no secuencialmente.
\item Tiempo de establecimiento de las tablas de encaminamiento: cuánto tiempo necesita Unity para pedir los datos a los routers (tablas de encaminamiento, IPs asociadas a cierta interfaz) con el fin de rellenar los carteles de la escena.
\end{enumerate}
Llevaremos a cabo esta prueba de forma automatizada sobre el juego compilado, insertando una instancia \texttt{StopWatch} en la clase \texttt{L1M1Handler} y exportando los distintos lapsos de tiempo en un fichero de texto que pueda ser consultado tras su escritura.
\end{itemize}

\section{Caso 1}
\subsection{Consumo de RAM}
La memoria bloqueada en el PC antes de que GNS3 y la máquina virtual de GNS3 VM sean abiertos es de 13652KB, los cuales serán descontados del total.

\begin{itemize}
\item Cuando el proyecto es abierto pero \textbf{ningún nodo es iniciado}, GNS3 aloja 131,7MB mientras que GNS3 VM 28,3125MB. Total: \textbf{160,01MB}.
\item \textbf{Arrancado} el proyecto, GNS3 152,1MB y GNS3 VM se mantiene igual. Total: \textbf{180,41MB}.
\end{itemize}

\subsection{Tiempo de arranque de los routers}
La siguiente tabla recoge el tiempo de arranque individual de cada nodo y la media del total.

\begin{table}[H]
\centering
\begin{tabular}{|l|l|l|l|l|l|}
\hline
\textbf{R1} & \textbf{R2} 	& \textbf{R3} 	& \textbf{R4} 	& \textbf{R5} 	& \textbf{Media}	\\ \hline
52,28s		& 52,38s		& 47,13s		& 51,26s		& 51,73s		& \textbf{50,956}	\\ \hline
\end{tabular}
\label{tab:t1}
\end{table}

Si ahora probamos a arrancar el proyecto completo (que incluye dos VPCs además de los routers), el resultado se dispara, necesitando de nada menos que \textbf{4min15s} para que todos los nodos sean completamente cargados. Es más de cuatro veces la media de lo que necesita cuando son ejecutados individualmente.

\subsection{Tiempo de establecimiento de la escena}
La tabla \ref{tab:t2} recoge los datos obtenidos tras la prueba:

\begin{table}[H]
\centering
\begin{tabular}{|c|c|c|}
\hline
\textbf{Establecimiento VPCs} & \textbf{Establecimiento routers} & \textbf{\begin{tabular}[c]{@{}c@{}}Establecimiento tablas\\ de encaminamiento\end{tabular}} \\ \hline
0,017s                        & 42,255s                          & 62,299s                                           \\ \hline
\end{tabular}
\label{tab:t2}
\end{table}

\section{Caso 2}
\subsection{Consumo de RAM}
De nuevo, la memoria bloqueada en el PC antes del experimento es de 13652KB.

\begin{itemize}
\item Cuando \textbf{está parado} el proyecto, la máquina virtual que contiene Xubuntu y donde está GNS3 instalado ocupa \textbf{261,61MB}.
\item Cuando se \textbf{inicia} todo el proyecto, la RAM usada pasa a ser de \textbf{790,61MB}.
\end{itemize}

\subsection{Tiempo de arranque de los routers}
La siguiente tabla recoge el tiempo de arranque individual de cada nodo y la media del total.

\begin{table}[H]
\centering
\begin{tabular}{|l|l|l|l|l|l|}
\hline
\textbf{R1} & \textbf{R2} 	& \textbf{R3} 	& \textbf{R4} 	& \textbf{R5} 	& \textbf{Media}	\\ \hline
60,26s		& 57,96s		& 58,21s		& 58,44s		& 57,95s		& \textbf{58,564}	\\ \hline
\end{tabular}
\label{tab:t3}
\end{table}

Como es de esperar, el tiempo que tardan todos los nodos en ser completamente inicializados también se ve incrementado. Hasta los \textbf{5min16s} concretamente.

\subsection{Tiempo de establecimiento de la escena}
La tabla \ref{tab:t4} recoge los datos obtenidos tras la prueba:

\begin{table}[H]
\centering
\begin{tabular}{|c|c|c|}
\hline
\textbf{Establecimiento VPCs} & \textbf{Establecimiento routers} & \textbf{\begin{tabular}[c]{@{}c@{}}Establecimiento tablas\\ de encaminamiento\end{tabular}} \\ \hline
0,017s                        & 46,246s                          & 66,294s                                           \\ \hline
\end{tabular}
\label{tab:t4}
\end{table}

\section{Conclusiones}
Como era de esperar, el caso 2, al estar basado íntegramente en una máquina virtual, ofrece peores resultados que el caso 1, ya que esta cuenta con los recursos de la máquina física de forma muy limitada y controlada.

\begin{figure}[H]
  \centering
  \includegraphics[scale=0.4]{imagenes/ram}
  \caption{Comparación consumo de RAM}
  \label{fig:ram}
\end{figure}

La figura \ref{fig:ram} evidencia la diferencia de consumo de RAM entre ambas opciones. Mientras que la primera apenas llega a los 200MBs, la segunda la cuadriplica. En consecuencia, será necesario contar con un PC de ciertas prestaciones para ser capaz de trabajar con redes grandes de usar este método.

\begin{figure}[H]
  \centering
  \includegraphics[scale=0.4]{imagenes/tiemposarranque}
  \caption{Tiempos de arranque de los nodos de forma individual (arriba) y en grupo (abajo)}
  \label{fig:tiemposarranque}
\end{figure}

El resultado también es negativo para el tiempo de arranque de los routers. Si bien en ninguno de los das casos la activación es automática como veníamos anunciando, el caso 2 sigue quedando detrás según se observa en la gráfica \ref{fig:tiemposarranque}.

\begin{figure}[H]
  \centering
  \includegraphics[scale=0.4]{imagenes/arranquejuego}
  \caption{Tiempos de arranque de las partes del juego relacionadas con el simulador}
  \label{fig:arranquejuego}
\end{figure}

Finalmente, la figura \ref{fig:arranquejuego} muestra que las diferencias a la hora de interactuar con el juego no son demasiado grandes entre los dos casos. El caso 1 es ligeramente más rápido que el segundo.

Aunque es evidente la derrota en rendimiento de la máquina virtual, las ventajas con ella son claras como se explicó al comienzo del capítulo. El hecho de aislar toda la configuración del simulador en una sola máquina virtual hace su uso extremadamente portable y sencillo. Problemas como su mayor consumo de RAM no debe serlo a día de hoy, con sistemas que cuentan fácilmente con 8GBs de RAM. Además, la sola instalación de GNS3 es ya de por sí poco recomendable para equipos de pocas prestaciones \cite{gnsweb}. Yendo aún a más: si el proyecto se encuentra inicialmente activo y configurado cuando el juego fuera a hacer uso de él, el caso 2 no apenas se ve perjudicado frente al segundo en cuanto a configuración del juego e interacción con este.

Consideramos entonces que es la opción correcta a seguir, ya que firmemente pensamos que las ventajas que ofrece las hacen ideales para el proyecto (acercar el juego al mayor número de personas) frente a los inconvenientes que es evidente que presenta.
\chapter{Conclusiones}\label{chap:Conclusiones}

fgxdh

\section{Descripción del problema}

huigk




\nocite{*}
\bibliographystyle{unsrt}
\bibliography{bibliografia/bibliografia}

%\appendix
%\input{apendices/manual_usuario/manual_usuario}
%%\input{apendices/paper/paper}
%\input{glosario/entradas_glosario}
% \addcontentsline{toc}{chapter}{Glosario}
% \printglossary
\chapter*{}
\thispagestyle{empty}

\end{document}
