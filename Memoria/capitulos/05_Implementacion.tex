\chapter{Integración}\label{chap:Integration}
Trazado el esquema del trabajo a realizar, el siguiente es lógicamente llevarlo a cabo. Las soluciones empleadas para todo el proceso será lo que componga este capítulo.

\section{Desarrollo de la API}
Si bien en la sección~\ref{sec:dis_api} ya se habló de la estructura con la que nuestra API contaría, en esta ocasión se contará con todo detalle el modo en el que esta ha sido desarrollada. Se describirán los puntos más importantes de cada una de las clases así como el modo en que fueron originados. Naturalmente, no se pretende explicar todo el contenido, pues no es su propósito. Se evitará así explicar en muchos casos métodos y propiedades privadas para centrarnos en aquellas públicas.

Para finalizar se explicará la relación existente entre las clases con ayuda de un diagrama UML así como la forma en la que estas han sido compiladas para ser unificadas como librería.

Todas estas clases conforman un único \texttt{namespace} o espacio de nombres. Conviene recalcar que la API ha sido enteramente desarrollada desde cero.

\subsection[GNS3sharp]{\href{https://github.com/aorestr/GNS3sharp/blob/master/gsn3sharp.cs}{GNS3sharp}}
\subsubsection{Constructor}
El constructor de \GNSCS~es sin duda uno de los elementos más importantes de toda a API. ¿Por qué? Tendrá la responsabilidad de conectarse al servidor de GNS3, recibir todos los datos de un cierto proyecto que solicitemos, procesarlos y guardarlos de tal forma que sean útiles. Su cabecera se muestra justo a continuación:
\begin{lstlisting}[language={[Sharp]C}, caption={Cabecera del constructor de \texttt{GNS3sharp}}, label={gnscs1}]
public GNS3sharp(string _projectID, string _host = "localhost", ushort _port = 3080)
\end{lstlisting}

Cosas que serán necesarias entonces para inicializarlo:
\begin{itemize}
\item \textbf{ID del proyecto}: cada proyecto de GNS3 tiene un ID asociado que el servidor guarda junto a su referencia. Más adelante, en la subsección \ref{subsec:aux}, veremos cómo no es necesario conocer el ID del proyecto sino que con el nombre del mismo será más que suficiente.
\item \textbf{Host}: la dirección del equipo donde el servidor está alojado. Por defecto se toma \texttt{localhost}; se supone que la mayor parte de las veces se encontrará en el mismo equipo desde el que se utiliza la API.
\item \textbf{Puerto}: además de la dirección del servidor, es necesario conocer el puerto en el que está montado. GNS3 determina el \texttt{3080} por defecto.
\end{itemize}

De acuerdo, ¿y qué hace exactamente con estos tres parámetros?
\begin{enumerate}
\item Crea la cadena de texto de la URI donde está el recurso asociado a los nodos del proyecto. En esta dirección existe únicamente un JSON con toda la información sobre él. Se instancia un cliente web y \textbf{se descarga el recurso como cadena de texto}.

\item Hay que \textbf{deserializar el JSON}. Este paso no es en absoluto trivial, ya que los métodos de \textit{Json.NET} son incapaces de extraer los datos que necesitamos directamente. Así que, valiéndonos de las herramientas en forma de clases que nos ofrece, lo hacemos manualmente.
\begin{lstlisting}[language={[Sharp]C}, caption={Deserialización de JSON}, label={gnscs2}]
// JSON array object
JArray jsonArray = JArray.Parse(json);
Dictionary<string,object> tempDict = new Dictionary<string, object>();

// Variables in which store the JSON info temporaly
string name; object value;        
if (jsonArray.HasValues){
    foreach (JObject jO in jsonArray.Children<JObject>()) {
        foreach (JProperty jP in jO.Properties()) {                
            name = jP.Name;
            value = (object)jP.Value;
            tempDict.Add(name,value);
            // The last key of every node
            if (jP.Name.Equals(lastKey)) {
                // If we do not copy the content of the dictionary into another
                // we will be copying by reference and erase the content once
                // we 'clear' the dict
                Dictionary<string, object> copyDict = new Dictionary<string, object>(tempDict);
                dictList.Add(copyDict);
                tempDict.Clear();
            }
        }
    }
}
\end{lstlisting}
Sin entrar mucho al detalle, lo que hace es parsear el objeto como objeto de tipo \texttt{JArray} y luego este se discretiza por cada elemento de esa cadena. Cada elemento representa a un nodo. Se recorre a su vez cada subelemento que lo conforma, que no son más que sus pares \textit{llave-valor}.

Los datos que se han extraído se guardan en una lista de diccionarios de par \texttt{<string,object>}. Se usa la clase genérica \texttt{object} porque el tipo del valor asociado a la clave varía.

\item Se hace lo mismo para los enlaces. El proceso es similar.

\item A partir de los objectos extraídos de la deserialización del JSON, se \textbf{crean las instancias representantes} de los nodos del proyecto. Existe un problema de cierta gravedad en esto: en el JSON no existe ningún parámetro que explicite de que tipo de nodo estamos hablando. Es decir, que no podemos conocer directamente con qué aparato concreto estamos tratando.

La solución que nosotros hemos tomado para sortear este problema es el añadir una etiqueta al nombre del nodo en el momento de la creación de la red en GNS3. Por ejemplo, si el nodo es un \textit{VPC} (nodo predefinido y propio del simulador), su nombre sería de la forma ``[VPC]NombreDelNodo''. Será necesario entonces definir una etiqueta por cada tipo de aparato a utilizar.

Sin embargo, sigue existiendo un problema de peso en todo esto: no conocemos el tipo de cada nodo antes del tiempo de ejecución (\textit{runtime}, en inglés) del código. En otras palabras, hasta el momento de la descarga y el análisis del JSON es imposible saber con qué tipo de nodos estamos lidiando, con lo que es a su vez imposible escribir el constructor que se va a utilizar. Recordemos que nuestra intención es usar una clase por cada modelo de nodo concreto. Por supuesto, se podría crear una sentencia condicional en la que, dependiendo de la etiqueta, llevara al constructor de una clase u otra. Sin embargo, requeriría de muchas líneas de código y sería necesario añadir más cada vez que se introduzca una nueva clase de modelo de aparato en la API.

La solución pasa entonces por lo que en programación se conoce como \textit{reflexión}. Poniéndolo en términos puramente técnicos: ``inspeccionar los metadatos y el código compilado en tiempo de ejecución se llama reflexión''\cite{csnutshell}.

\begin{lstlisting}[language={[Sharp]C}, caption={Instanciación de los nodos}, label={gnscs3}]
Type classType; int i = 0;
try{
    foreach(Dictionary<string, object> node in JSON){
        try{
            Console.WriteLine($"Gathering information for node #{i}... ");
            // Assign a class or another depending on the node
            classType = Aux.NodeType(node["name"].ToString());
            listOfNodes[i] = (Node)Activator.CreateInstance(
                classType,
                node["console_host"].ToString(), 
                ushort.Parse(node["console"].ToString()), 
                node["name"].ToString(),
                node["node_id"].ToString(),
                GetNodeListOfPorts(node)
            );
            nodesByName.Add(listOfNodes[i].Name, listOfNodes[i]);
            nodesByID.Add(listOfNodes[i].ID, listOfNodes[i]);
            i++;
        } catch(Exception err1){
            Console.Error.WriteLine(
                "Impossible to save the configuration for the node #{0}: {1}", 
                i.ToString(), err1.Message
            );
        }
    }
} catch(Exception err2){
    Console.Error.WriteLine(
        "Some problem occured while saving the nodes information: {0}",
        err2.Message
    );
    listOfNodes = null;
}
\end{lstlisting}
Con esto en mente pues, tomamos el tipo de la clase asociada al aparato mediante la etiqueta que ya hemos mencionado y usamos el método \texttt{Activator.CreateInstance()} para instanciar el objecto. Únicamente requiere los metadatos del tipo representante (la clase) del nodo (representados a su vez por la clase \texttt{System.Type}) y los parámetros que el constructor de la clase necesita.

Añadimos la instancia a una lista donde se guardan todos los objetos representantes de los nodos. Esta propiedad se verá más adelante.

La obtención de las interfaces que cada nodo posee así como de cuáles están libres y cuáles están usadas por cierto enlace requiere de más trabajo que no se considera necesario explicar aquí.

\item Es ahora el momento de instanciar los enlaces a través de la información extraída previamente. Aunque el problema que vimos con los nodos no se va a dar aquí, la obtención de los enlaces y sus parámetros tampoco es fácil. La principal dificultad aparece, una vez más, cuando se trata con las interfaces de los nodos.

\begin{lstlisting}[language={[Sharp]C}, caption={Matcheo entre los enlaces y las interfaces de los nodos}, label={gnscs4}]
List<Dictionary<string,object>> dictList = null;
try{
    dictList = DeserializeJSONList(nodesJSON, "port_number");
} catch (Exception err){
    Console.Error.WriteLine(
        "Some problem occured while trying to gather information about the nodes connect to the link: {0}",
        err.Message
    );
}

if (dictList.Count > 0){
    // Iterates through the JSON dictionary
    foreach (Dictionary<string, object> nodeTemp in dictList){
        // Iterates through the nodes the link connects
        foreach (Node node in link.Nodes){
            if (node != null && node.ID.Equals(nodeTemp["node_id"].ToString())){
                // Search for the port that matches the found one
                var foundPort = node.Ports.Where(
                    x => (
                        x["adapterNumber"].ToString() == nodeTemp["adapter_number"].ToString() &&
                        x["portNumber"].ToString() == nodeTemp["port_number"].ToString()
                    )
                );
                // If exists, add the link into the key "link" of the ports list
                // of dictionaries of the node
                if (foundPort.Count() > 0) foundPort.First()["link"] = link;
            }
        }
    }
}
\end{lstlisting}

Una vez localizada la sección relacionada con los nodos en el diccionario que contiene la información de los enlaces extraída del JSON, se deserializa y se hace un barrido por entre esos nodos. Hecho esto, se busca entre los nodos que ya tenemos almacenados aquel que posea la misma ID que el del nodo sobre el que estamos barriendo. Si se encuentra, se busca la interfaz del mismo que dicte el nodo del barrido. De encontrarse, el puerto (usado aquí como sinónimo de interfaz) del nodo que teníamos almacenado guardará la información del enlace que contiene todos esos nodos que se están analizando. Es bastante confuso, sí.

El método principal que se encarga de la extracción de los enlaces, \texttt{GetLinks()}, se verá ayudado por un par de funciones declaradas localmente. Estas funciones locales aparecen por vez primera en C\#7.

Se guardan todos estos enlaces como lista en otra de las propiedades que se verán más adelante.

\item Finalmente, se añade a cada nodo la información de los enlaces a los que está conectado.

\end{enumerate}

\subsubsection{Propiedades}
\begin{itemize}
\item \texttt{ProjectID}, \texttt{Host} y \texttt{Port}: propiedades que se toman directamente de los parámetros que el constructor necesitaba.
\item \texttt{NodesJSON} y \texttt{LinksJSON}: diccionarios que contienen los JSON descargados desde el servidor una vez han sido parseados. Por lo general esta propiedad no tendría porque ser usada por un usuario que no esté desarrollando la API, pero se mantiene como \texttt{public} por si hay algún caso en el que sí.
\item \texttt{Nodes} y \texttt{Links}: posiblemente las propiedades más importantes de la clase. Son listas que contienen los objetos que representan los nodos y enlaces, respectivamente, del proyecto.
\end{itemize}

\subsubsection{Métodos}
La mayor parte de los métodos de esta clase son privados, pues son usados internamente como subrutinas de otros métodos más grandes. Sin embargo podemos encontrar algunos accesibles desde fuera de la clase:
\begin{itemize}
\item \texttt{StartNode()} y \texttt{StopNode()}: activan/desactivan un nodo del proyecto. Esto se consigue haciendo uso del método POST de REST hacia la URI correspondiente al nodo. En el código se lleva a cabo mediante la clase \texttt{System.Net.Http.HttpClient}, que provee las herramientas suficientes para enviar y recibir datos de un recurso web.

\begin{lstlisting}[language={[Sharp]C}, caption={Activación/desactivación de un nodo}, label={gnscs5}]
// First part of the URL
string URLHeader = $"http://{host}:{port}/v2/projects/{projectID}/nodes";

// Pack the content we will send
ByteArrayContent byteContent = null;
try{
    string content = JsonConvert.SerializeObject(new Dictionary<string, string> { { "", "" } });
    byteContent = new ByteArrayContent(System.Text.Encoding.UTF8.GetBytes(content));
    byteContent.Headers.ContentType = new MediaTypeHeaderValue("application/json");
} catch(JsonSerializationException err){
    Console.Error.WriteLine("Impossible to serialize the JSON to send it to the API: {0}", err.Message);
}

if (byteContent != null){
    try{
        responseStatus = HTTPclient.PostAsync(
            $"{URLHeader}/{node.ID}/{status}", byteContent
        ).Result.IsSuccessStatusCode;
    } catch(HttpRequestException err){
        Console.Error.WriteLine("Some problem occured with the HTTP connection: {0}", err.Message);
        responseStatus = false;
    } catch(Exception err){
        Console.Error.WriteLine("Impossible to {2} node {0}: {1}", node.Name, err.Message, status);
        responseStatus = false;
    }
} else{
    responseStatus = false;
}
\end{lstlisting}

\item \texttt{StartProject()} y \texttt{StopProject()}: activan/desactivan todos los nodos del proyecto. Se intentó paralelizar el activado/desactivado de los nodos sin resultado.
\item \texttt{SetLink()}: pasándole los objetos representantes de dos nodos del proyecto, es capaz de descubrir cuáles de sus interfaces están vacías y, de haber, crea un enlace entre ellas. También se consigue mediante POST. Actualiza \texttt{Links} y otra serie de parámetros tras la inserción. Es un método de cierta longitud (algo más de 100 líneas sin contar otras definidas fuera de las que hace uso).
\item \texttt{EditLink()}: método polimórfico, pues dependiendo de si su parámetro es un \LINK~o dos \NODE~su comportamiento varía. Se parece a \texttt{SetLink()} pero este hace PUT y no POST a la URI. Ambas formas del método llaman a un método interno de \LINK. Tal y como su nombre indica, edita un enlace, permitiendo hacer variar los parámetros del mismo.
\item \texttt{RemoveLink()}: con la misma base polimórfica que el anterior. Elimina un enlace del proyecto de GNS3 con un DELETE e inmediatamente a su representante objeto.
\item \texttt{GetNodeByName()} y \texttt{GetNodeByID()}: dado un nombre o un identificador, respectivamente, devuelve el objecto representante de tal nodo.
\end{itemize}

\subsection{Node}
\subsubsection{Constructor}
El constructor principal de \NODE~solo es llamado desde \GNSCS. Es bastante sencillo: asigna parámetros básicos que la instancia de \GNSCS~toma del servidor. Entre ellos se encuentra la dirección del nodo. Gracias a ella y mediante otro método interno de la clase, se crea un cliente TCP y se establece un flujo de conexión para el envío y recepción de mensajes.

\begin{lstlisting}[language={[Sharp]C}, caption={Establecimiento de la conexión con el nodo}, label={node2}]
protected (TcpClient Connection, NetworkStream Stream) Connect(int timeout = 10000){
    // Network endpoint as an IP address and a port number
    IPEndPoint address = new IPEndPoint(IPAddress.Parse(this.consoleHost),this.port);
    // Set the socket for the connection
    TcpClient newConnection = new TcpClient();
    // Stream used to send and receive data
    NetworkStream newStream = null;
    try{
        newConnection.Connect(address);
        newStream = newConnection.GetStream();
        newStream.ReadTimeout = timeout; newStream.WriteTimeout = timeout;
    } catch(Exception err){
        Console.Error.WriteLine("Impossible to connect to the node {0}: {1}", this.name, err.Message);
        newConnection = null;
    }
    return (newConnection, newStream);
}
\end{lstlisting}

Especial atención a este método, que devuelve una tupla (recordemos, también nuevo en C\#7) en lugar de una simple variable.

Para que solo clases que pertenecen a esta librería puedan hacer uso del constructor, se ha aplicado el modificador de clase \texttt{internal}, el cual solo permite llamadas desde el espacio de nombres donde esté definido. Encapsulamiento en estado puro. La API no permite la creación de nodos nuevos así que se ha optado por ocultar este método del desarrollador final.

No obstante, sí que se incluye un constructor-clonador. Se trata de un constructor de clase cuyo parámetro de entrada es otro \NODE, de forma que se replica enteramente en una nueva instancia.

\begin{lstlisting}[language={[Sharp]C}, caption={Clonador de nodos}, label={node2}]
public Node(Node clone){
    this.consoleHost = clone.ConsoleHost; this.port = clone.Port;
    this.name = clone.Name; this.id = clone.ID; this.ports = clone.Ports;
    this.tcpConnection = clone.TCPConnection; this.netStream = clone.NetStream;
}
\end{lstlisting}

\subsubsection{Propiedades}
\begin{itemize}
\item \texttt{ConsoleHost} y \texttt{Port}: dirección y puerto donde el nodo está ubicado. Gracias a estos datos podremos establecer una conexión con el aparato.
\item \texttt{Name} y \texttt{ID}: nombre e identificador único del nodo. El nombre debería comenzar por \textit{[$<$EtiquetaDelNodo$>$]} para que \GNSCS~sea capaz de construir el objeto con la clase asociada a tal aparato.
\item \texttt{Ports}: interfaces que posee el nodo. Es un diccionario de tres llaves: ``adapterNumber'', ``portNumber'' y ``link''. Los dos primeros son parámetros que GNS3 asigna a las interfaces. El último guarda la referencia del enlace asociado a la interfaz en caso de que esté siendo utilizada y \texttt{null} si no.
\item \texttt{LinksAttached}: lista de \LINK. Referencias a los que el nodo está conectado.
\end{itemize}

\subsubsection{Métodos}
Esta clase destaca por sus dos métodos principales: \texttt{Send()} y \texttt{Receive()}.

\begin{itemize}
\item \texttt{Send()}: haciendo uso del flujo establecido durante la creación del objeto, se envía una cadena de texto previamente convertida a bytes. Antes de enviar cualquier mensaje, comprueba que es posible escribir en el canal.

\begin{lstlisting}[language={[Sharp]C}, caption={Envío de mensajes a un nodo}, label={node3}]
byte[] out_txt = Encoding.Default.GetBytes($"{message}\n");
this.netStream.Write(buffer: out_txt, offset: 0, size: out_txt.Length);
this.netStream.Flush();
\end{lstlisting}

\item \texttt{Receive()}: algo más complejo que \texttt{Send()}, hace también uso del canal establecido, aunque necesita de pasos adicionales para gestionar correctamente la información que se recibe.

\end{itemize}

\subsubsection{Destructor}
Esta es la única clase de la librería que hace uso de un destructor personalizado. Los destructores ayudan a definir las sentencias que serán ejecutadas junto antes de que el objeto en cuestión sea destruido (se desreferencie).

Únicamente se encarga de cerrar la conexión establecida con el nodo.

\subsection{Link}
\subsubsection{Constructor}
\subsubsection{Propiedades}
\subsubsection{Métodos}

\subsection{Herederos de Node}

\subsection{Clases auxiliares}\label{subsec:aux}

\subsection{Compilación}

\section{Desarrollo del videojuego}

\subsection{El proyecto de GNS3}

\subsubsection{Aparatos usados}

\subsubsection{Despliegue}

\subsection{El proyecto de Unity}

\subsubsection{Materiales}

\subsubsection{Creación del juego}
