\chapter{Analisis tecnológico}\label{chap:Analisis}

fgxdh

\section{Simulador de redes: GNS3}
hay que comentar qué mecanismos nos interesan de GNS3. Si no se ha explicado cómo se organiza y funciona antes, aquí es el lugar.

GNS3 es usado ampliamente como método de entrenamiento para los exámenes de Cisco. Tal es así, que cuenta con su \MYhref{https://academy.gns3.com/p/the-complete-networking-fundamentals-course-your-ccna-start}{propia academia online} de cursos en los que se utiliza el simulador para estudiar, convirtiendo la formación en más accesible. Sin embargo, las imágenes de las máquinas de Cisco, aunque pueden ser encontradas fácilmente en internet, requieren de licencia para ser usadas. Siendo así se optó por hacer uso de software libre para el proyecto. 

\subsubsection{Nodos}
Una figura esquemática que muestre todos estos conceptos, es fundamental al comienzo de un capítulo o sección importante.

Cada elemento de una red está representado en GNS3 por un elemento llamado \textbf{nodo}. Estos nodos, que pueden ser desde un router a un switch, no son más que virtualizaciones de aparatos reales. De normal, estas virtualizaciones se realizan en base a imágenes de los sistemas operativos que se integran en los aparatos. Así, podemos tener varios routers distintos de Cisco montados sobre la misma estructura, permitiéndonos jugar con ellos con verdadera facilidad.

\subsubsection{Enlaces}

\subsubsection{La API de GNS3}
Una API (que será explicada más adelante)
La primera vez que aparezca un acrónimo, debes indicar cuál es su significado. De hecho, en los títulos o como primera palabra de la frase, (o en el abstract) hay que evitar las abreviaturas.


\section{Motor de videojuegos: Unity}
realmente esta elección no era tan complicada