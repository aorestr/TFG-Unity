\chapter{Introducción}\label{chap:Intro}
Este primer capítulo introductorio sirve de acercamiento tanto al problema con el que se pretende lidiar como al modo con el que se decidió resolverlo. Al final del mismo se listarán los distintos capítulos del documento así como un breve resumen de cada uno de ellos.

\section{Introducción}
El siguiente trabajo se propone desarrollar un \textbf{videojuego capaz de integrar un emulador de redes} (o simulador, usaremos ambas palabras indistintamente) en sus mecánicas; esto es, capaz de \textbf{verse afectado} por la configuración establecida en una red virtualizada y, a la misma vez, tener el poder de modificarla desde el propio juego. La finalidad de todo esto no reside en el entretenimiento puro que brinda el mundo del videojuego, sino en las puertas que abre para el mundo docente. Si consideramos a los videojuegos un medio para acceder al conocimiento, de hacerlo más atractivo al público, la integración de redes reales en ellos puede permitir crear un nuevo camino docente en el ámbito de las redes. Finalmente no es más que una forma de facilitar el aprendizaje para un amplio número de personas.

Se analizará una serie de tecnologías en busca de las más adecuadas para el trabajo. Decididas estas, se creará sobre ellas una solución que ejemplifique el fin del proyecto.

\subsection{Motivación}
En realidad, la idea principal.. quizás sea aquí donde puedes hablar de las deficiencias de juegos parecidos, o las fortalezas de usar un emulador de red real para enseñar estos temas.

En realidad, pon Motivación, y dentro, un párrafo con la motivación personal, si quieres (no es necesario poner la motivación personal).

En "motivación" pondrás qué deficiencias hay, o qué oportunidades de mejora, que hacen que este proyecto sea necesario.

\subsection{Nuestro papel}
 
\subsection{Objetivos}
Aclarado lo anterior, los objetivos que el proyecto se propone son claros:
\begin{itemize}
\item Posibilitar la \textbf{interacción directa entre un videojuego y un simulador de redes}. Aquí reside nuestro papel fundamental del proceso: demostrar que es posible y de qué forma lo es.
\item \textbf{Desarrollar un videojuego} que haga uso de tal posibilidad; que sea capaz, asimismo, de encontrar un uso didáctico para ella y aplicarlo.
\end{itemize}

\section{Estructura del trabajo}
Para finalizar el capítulo, se expondrá a modo de adelanto una lista que recoge cada capítulo a tratar en el proyecto. Junto a todos ellos, se incluye un pequeño resumen que pretende dar a conocer qué podrá estudiarse en sus respectivas páginas:
\begin{itemize}
\item \textbf{Capítulo \ref{chap:Intro}: Introducción}. En el actual capítulo, como ya se ha podido ver, se han abordado los distintas causas que han motivado a existir a este proyecto a la vez que la forma en la que se intentará llevar a buen puerto su resolución.
\item \textbf{Capítulo \ref{chap:Plan}: Planificación}. Una guía del trabajo llevado a cabo, concerniente tanto a la manera en que se ha distribuido temporalmente como a los utensilios empleados.
\item \textbf{Capítulo \ref{chap:ArtState}: Estado del arte}. Las distintas tecnologías barajadas sobre las que trabajar son aquí listadas.
\item \textbf{Capítulo \ref{chap:Analisis}: Análisis tecnológico}. Tras citar algunos ejemplos de tecnologías que podrían emplearse para el proyecto en el anterior capítulos, aquí se hace un análisis más en profundidad de las que han resultado escogidas.
\item \textbf{Capítulo \ref{chap:Design}: Diseño}. Se determina la forma en la que se llevará a cabo la solución del problema inicial.
\item \textbf{Capítulo \ref{chap:Integration}: Implementación}. Trazado un camino a seguir en el capítulo anterior, en este se describe la manera en la que este camino toma forma. Se muestra qué se ha creado finalmente y cómo se ha logrado.
\item \textbf{Capítulo \ref{chap:Pruebas}: Pruebas}. En este capítulo el resultado final es mostrado y se añade una valoración del mismo.
\item \textbf{Capítulo \ref{chap:Conclusiones}: Conclusiones}. Para finalizar el proyecto, se ha hecho un resumen del trabajo realizado junto a una valoración del mismo. Además, algunas propuestas de trabajos futuros sobre él son dadas.
\end{itemize}
