\chapter{Introducción}\label{chap:Intro}
Este primer capítulo introductorio sirve de acercamiento tanto al problema con el que se pretende resolver como al modo con el que se decidió resolverlo. Al final del mismo se listarán los distintos capítulos del documento así como un breve resumen de cada uno de ellos.

\section{Introducción}
El siguiente trabajo se propone desarrollar un \textbf{videojuego capaz de integrar un simulador de redes} en sus mecánicas; esto es, capaz de \textbf{verse afectado} por la configuración establecida en una red virtualizada y, a la misma vez, tener el poder de modificarla desde el propio juego. La finalidad de todo esto no reside en el entretenimiento puro que brinda el mundo del videojuego, sino en las puertas que abre para el mundo docente. Si consideramos a los videojuegos un medio para facilitar el aprendizaje, de hacerlo más atractivo al público, la integración de redes reales en ellos puede permitir crear un nuevo camino docente en el ámbito de las redes. Finalmente no es más que una forma de facilitar el aprendizaje para el mayor número de personas posible.

Se analizará una serie de tecnologías para seleccionar las más adecuadas para el trabajo. Como resultado de todo lo anterior, se proporcionará una biblioteca para que los desarrolladores puedan crear sus propios juegos sobre el simulador de red, y un juego de ejemplo de uso.

\subsection{Motivación}
\subsubsection{Los videojuegos y la educación}
Contrariamente al modelo educativo tradicional, en que el que la atención gira en torno al profesor, desde principios del nuevo milenio aparece el \textit{aprendizaje centrado en el estudiante}. Aunque el origen del concepto puede datarse a principios del siglo pasado, no ha sido hasta este que ha comenzado a tomar forma. Carl Rogers, en su libro ``Freedom to Learn for the 80s'', describe el cambio de poder del profesor experto al alumno, impulsado por la necesidad de un cambio en el entorno tradicional en el que, en esta "llamada atmósfera educativa, los estudiantes se vuelven pasivos, apáticos y aburridos"\cite{studentcentered}.

R.M Harden y Joy Crosby, en un artículo realizado en el 2000, describen las estrategias de aprendizaje centradas en el ``maestro'' desde el enfoque de \textit{el maestro que transmite el conocimiento}, del experto al principiante. En contraste, describen el aprendizaje centrado en el estudiante como aquel enfocado en el aprendizaje de los estudiantes y ``lo que los estudiantes hacen para lograrlo, en lugar de lo que hace el maestro''. Esta definición enfatiza el concepto del estudiante \textit{haciendo} \cite{goodteacher}.

Es aquí donde los videojuegos pueden ayudar. Este mismo año, un informe salía a la luz dando una serie de datos bastante interesantes al respecto. El informe, desarrollado por \href{https://triseum.com/}{Triseum} (empresa dedicada de lleno a los videojuegos educativos), es un proyecto europeo con el propósito de probar y evaluar el aprendizaje basado en juegos con dos videojuegos educativos (ambos juegos creados por la propia empresa del estudio). De entre los 20 profesores que llevaron a sus aulas los juegos, ninguno de ellos tuvo experiencias negativas con ellos. Algunos de los más satisfechos con la prueba contaban que ``los estudiantes estaban muy motivados y mucho mejor preparados para las siguientes asignaturas y conceptos de la clase, construidos mediante el contenido y los conceptos tratados en el juego''\cite{triseum}.

Cientos de estudios se han llevado a cabo con el fin de demostrar el potencial didáctico de los videojuegos. Algunos de ellos se han centrado en exponer el uso de juegos comerciales en escuelas \cite{gamesinclass}, y otros tantos  el de juegos nacidos por y para la educación \cite{edugamesinclass}. Todos ellos, como el realizado por Kennedy-Clark y Thompson en 2011, concluyen que los beneficios de usar los videojuegos como método de aprendizaje funciona. Este último, asevera que ``como se encontró en muchos otros estudios, el mundo virtual fue motivador, interesante y atractivo para los estudiantes''\cite{deathrome}.

Y es que la virtualización es una herramienta enormemente versátil. Empresas como Labster se han dado cuenta de ello. Esta compañía se encarga de desarrollar laboratorios virtuales y propuestas lúdicas para la enseñanza de materias relacionadas con la ciencia, muchas de ellas llevadas a cabo en forma de videojuego\cite{labster}. ``Al integrar simulaciones en laboratorios virtuales permitimos que los alumnos tengan acceso ilimitado y económico y que practiquen todo lo que quieran hasta que aprendan los conceptos perfectamente'', dice su fundador, el danés Michael Bodekaer.	

\subsubsection{Enfoque del trabajo}
Es entonces nuestra misión llevar esto último al terreno de las telecomunicaciones. A sabiendas de que los videojuegos suponen un portal educativo de gran riqueza, nos proponemos desarrollar un ejemplo de \textbf{juego que permita aprender telemática}.

La telemática ``describe los procesos de transmisión y gestión de informaciones digitales así como los servicios y aplicaciones que se apoyan en ellos''\cite{telematica}. Entre esos servicios se encuentran las redes. La infraestructura asociada a estas suele ser cara y costosa, requiriendo un desembolso importante así como un despliegue inicial no apto para cualquiera. Esto recuerda directamente a las palabras anteriormente expuestas por el fundador de Labster. Su solución fue virtualizar laboratorios e insertarlos en juegos para, así, acercar estos a los estudiantes. Nosotros aquí pretendemos algo similar.

El fin que se propone este trabajo es el de crear un videojuego que integre una red virtualizada. Por suerte para nosotros, a diferencia de Bodekaer, no tenemos que encargarnos de crear un sistema de virtualización de redes, pues, como veremos con detalle en el capítulo \ref{chap:ArtState}, ya existen varios. El objetivo principal pasa por enlazar tal simulador de redes (una herramienta de virtualización de las mismas) con un videojuego.

Precisamente aquel es el primordial problema con el que se habrá de lidiar. A día de hoy, no existe forma directa de unir un videojuego con un simulador de redes. Hay que considerar que estos simuladores están concebidos como finalidad en sí mismos (como una aplicación cualquiera) y no como herramienta para ser usada por utilidades externas. Se requiere por consiguiente desarrollar una infraestructura que haga de nexo entre la red virtual y un videojuego.
 
\subsection{Objetivos}
Aclarado lo anterior, los objetivos que el proyecto se propone son los siguientes:
\begin{itemize}
\item Posibilitar la \textbf{interacción directa entre un videojuego y un simulador de redes}. Aquí reside nuestro papel fundamental del proceso. Con tal propósito crearemos una librería desde la que poder realizar la interacción.
\item \textbf{Desarrollar un videojuego} que haga uso de tal librería; que sea capaz, asimismo, de encontrar un uso didáctico para ella y aplicarlo.
\end{itemize}

\section{Estructura del trabajo}
Para finalizar el capítulo, se expondrá a modo de adelanto una lista que recoge cada capítulo a tratar en el proyecto. Junto a todos ellos, se incluye un pequeño resumen que pretende dar a conocer qué podrá estudiarse en sus respectivas páginas:
\begin{itemize}
\item \textbf{Capítulo \ref{chap:Intro}: Introducción}. En el actual capítulo, como ya se ha podido ver, se han abordado los distintas causas que han motivado a existir a este proyecto a la vez que la forma en la que se intentará llevar a buen puerto su resolución.
\item \textbf{Capítulo \ref{chap:Plan}: Planificación}. Una guía del trabajo llevado a cabo, concerniente tanto a la manera en que se ha distribuido temporalmente como a los utensilios empleados.
\item \textbf{Capítulo \ref{chap:ArtState}: Estado del arte}. Las distintas tecnologías barajadas sobre las que trabajar son aquí listadas.
\item \textbf{Capítulo \ref{chap:Analisis}: Análisis tecnológico}. Tras citar algunos ejemplos de tecnologías que podrían emplearse para el proyecto en el anterior capítulos, aquí se hace un análisis más en profundidad de las que han resultado escogidas.
\item \textbf{Capítulo \ref{chap:Design}: Diseño}. Se determina la forma en la que se llevará a cabo la solución del problema inicial.
\item \textbf{Capítulo \ref{chap:Integration}: Implementación}. Trazado un camino a seguir en el capítulo anterior, en este se describe la manera en la que este camino toma forma. Se muestra qué se ha creado finalmente y cómo se ha logrado.
\item \textbf{Capítulo \ref{chap:Pruebas}: Pruebas}. En este capítulo el resultado final es mostrado y se añade una valoración del mismo.
\item \textbf{Capítulo \ref{chap:Conclusiones}: Conclusiones}. Para finalizar el proyecto, se ha hecho un resumen del trabajo realizado junto a una valoración del mismo. Además, algunas propuestas de trabajos futuros sobre él son dadas.
\end{itemize}
