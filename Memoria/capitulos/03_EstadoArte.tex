\chapter{Estado del arte}\label{chap:ArtState}

El estado del arte podría definirse como el nivel de desarrollo de un ámbito concreto, generalmente relacionado con el mundo técnico-científico. Este capítulo se encarga de hacer un listado de aquellas tecnologías que podrían sernos útiles así como de dar una serie de apuntes acerca de cada una de ellas. Dividiremos las tecnologías en dos grandes secciones: simuladores/emuladores de red y motores de videojuego.

\section{Simuladores y emuladores de red}\label{sec:simuladoremulador}
Un simulador de red es un software que predice el comportamiento de una red de datos. En los simuladores, la red se modela con dispositivos, enlaces, aplicaciones, etc. y se configura su rendimiento. La mayoría de los simuladores utilizan la simulación de eventos discretos (modelado de sistemas en los que las variables de estado cambian en momentos discretizados de tiempo)\cite{wiki:Network_simulation}. El comportamiento de la red y las diversas aplicaciones y servicios que soporta pueden ser observados en un laboratorio de pruebas; varios atributos del entorno también pueden ser modificados de manera controlada para evaluar cómo se comportarían la red bajo diferentes condiciones.

Sin embargo, conviene no confundir ``simuladores'' con ``emuladores'' de red. Un simulador puede realizar tareas replicadas para demostrar el comportamiento de una red y sus componentes, mientras que un emulador puede copiar el comportamiento de una red para reemplazarla funcionalmente \cite{emsim}.

En un nivel básico, un simulador de red utiliza fórmulas matemáticas para crear un modelo teórico y totalmente virtual de una red. Los simuladores son soluciones software. Si bien se utilizan principalmente con fines de investigación y educación, también pueden actuar como herramientas de prueba cruciales en el diseño y desarrollo de una red. Nosotros mostraremos ejemplos de ambos indistintamente. Solo citaremos aquellos gratuitos y/o de código abierto.

\subsection{Mininet}
\textbf{Mininet} es un emulador de red que crea una red de hosts virtuales, switches, controladores y enlaces \cite{mininet}. Utiliza la virtualización basada en procesos de forma que es capaz de ejecutar muchos nodos (según la web, ha logrado levantar incluso 4096 a la vez) en un único núcleo del sistema operativo. Desde su versión 2.2.26, Linux soporta \texttt{namespace}s de red, una característica de virtualización ligera que proporciona a los procesos individuales interfaces de red separadas, tablas de enrutamiento y tablas ARP (del inglés, \textit{Address Resolution Protocol}). Mininet puede crear switches OpenFlow en el kernel o en el espacio de usuario, controladores para controlar los conmutadores y hosts para comunicarse a través de la red simulada. Se encarga de conectarlos usando pares de ethernet virtual (\texttt{veth}).

Mininet presenta una serie de ventajas muy interesantes en el mundo de la virtualización de redes:
\begin{itemize}
\item Capacidad de alojamiento de miles de dispositivos en una misma red. 
\item Asegura un arranque rápido de los nodos.
\item Tiene una instalación muy sencilla, pues es posible descargarse desde su web máquinas virtuales con todo preparado para que tan solo sea necesario instalarla en un hipervisor. También admite instalación nativa para Linux.
\item Cuenta con una \MYhref{http://mininet.org/api/annotated.html}{API} (del inglés, \textit{Application Programming Interface} desde la que poder interactuar con la red y los dispositivos desplegados en ella. Sin embargo, está escrita en Python y no hemos dado con ningún motor de videojuegos que admita a este como lenguaje de scripting nativo.
\item También posee una interfaz de usuario desde la que construir y gestionar la red.
\end{itemize}

Los últimos commits del \MYhref{https://github.com/mininet/mininet}{repositorio de Mininet} son anteriores al último mes, con lo que es claro que el proyecto sigue siendo mantenido.

\subsection{Common Open Research Emulator (CORE)}
\textbf{CORE} es una herramienta para construir redes virtuales. Como emulador, construye una representación de una red telemática real que funciona en tiempo real, a diferencia de la simulaciones de red, en las que, como ya vimos, se utilizan modelos abstractos. La emulación en tiempo real puede conectarse a redes físicas y routers. Proporciona un entorno para ejecutar aplicaciones y protocolos reales, aprovechando las ventajas de la virtualización proporcionada por Linux o FreeBSD, únicos sistemas operativos donde está disponible. 

Algunas de sus características clave son:
\begin{itemize}
\item Es eficiente y escalable.
\item Ejecuta aplicaciones y protocolos reales sin modificaciones.
\item Cuenta con una interfaz gráfica de usuario fácil de usar.
\item Es altamente personalizable.
\end{itemize}

CORE se utiliza típicamente para la investigación de redes y protocolos, demostraciones, pruebas de aplicaciones y plataformas, evaluación de escenarios de redes, estudios de seguridad y aumento del tamaño de las redes físicas de prueba. Permite la inclusión de otras tecnologías de emulación de red como EMANE y ns-3 \cite{core}.

Sus características fundamentales son similares a las del emulador previamente citado. Al igual que aquel, CORE puede ser usado a través del GUI (la interfaz de usuario) o scripting mediante Python. Además, cuenta con una API propia. El demonio de Linux que CORE levanta escucha en un socket TCP para los mensajes de la API que lleguen de otra aplicación local o remota. La interfaz de usuario de CORE se conecta localmente a este demonio y utiliza la API para instanciar topologías. CORE también actuará como una ``servidor de emulación'' que escucha conexiones TCP de otro sistema. En el momento de la conexión, el otro sistema transmite mensajes al demonio de CORE, el cual puede controlar la la emulación en tiempo real \cite{coreapi}.

\subsection{GNS3}
GNS3 es un simulador gráfico de red multiplataforma \href{https://github.com/GNS3}{open-source} que se ejecuta en Windows, macOS y Linux \cite{bookgns}. Esto ya presenta una cierta ventaja respecto a sus anteriores competidores, cuya instalación solo es posible en Linux (y FreeBSD en el caso de CORE). Permite la virtualización de dispositivos reales tales como routers Cisco (para lo cual está especialmente preparado) y conectividad con otras redes reales como internet. Según su web, algunas de las marcas que utilizan este software son at\&t, Intel, NASA o Verizon \cite{gnsweb}.

La aplicación crea un servidor en la máquina en la que es lanzado al cual se puede acceder mediante una API REST. 

Cuenta con \MYhref{https://docs.gns3.com/}{documentación online}, \MYhref{https://gns3.com/community}{foros muy activos} e incluso una \MYhref{https://academy.gns3.com/}{academia} donde se imparten cursos.  

En el capítulo \ref{chap:Analisis} hablaremos con más detenimiento de este simulador.

\subsection{Más simuladores/emuladores}
Existen más aplicaciones que virtualizan redes que no explicaremos aquí. Por citar algunas: Cloonix, Netkit, Shadow, VNX...


\section{Motores de juego}
Un motor de juego proporciona multitud de características que son útiles a través de muchos juegos diferentes, por lo que un juego implementado con ese motor obtiene todas esas características junto a elementos artísticos personalizados y código de juego específico para él \cite{unityinaction}. 

Dividiremos cada uno de los que veremos en tres secciones fundamentales: plataformas hacia donde los juegos pueden ser exportados, características, lenguajes de programación que usan y precio.

\subsection{Unreal Engine}
Con una lista inmensa de juegos desarrollados con él \cite{wiki:unrealgames}, podemos asegurar que \textbf{Unreal Engine} (y más concretamente Unreal Engine 4, su última versión), de la compañía Epic Games, es uno de los motores de juego más usados de la historia. Es por eso, y por las características que veremos a continuación, que aparece en las listas de motores más famosos.
\begin{itemize}
\item \textbf{Plataformas soportadas}: Windows PC, PlayStation 4, Xbox One, macOS, iOS, Android, AR, VR, Linux, SteamOS y HTML5. El editor tiene soporte en Windows, Linux y macOS \cite{eulaunreal}.
\item \textbf{Características}: renderización fotoreal en tiempo real, efectos de calidad de película de post-procesamiento, un flexible editor de materiales, abundantes herramientas de animación...\cite{unrealfeatures} Unreal no ha llegado a ser un motor tan importante en vano. Quizá, de entre todas sus prestaciones, la más interesante sea los ``blueprints''. Se trata de una forma visual y amigable de programar. Es posible crear rápidamente prototipos y contenido interactivo sin tocar una línea de código. Puede ser utilizad para construir comportamientos e interacciones de objetos, modificar interfaces de usuario, ajustar controles de entrada y mucho más.
\item \textbf{Lenguajes de scripting}: aunque anteriormente era posible usar UnrealScript como lenguaje para desarrollar en Unreal, a partir de la versión 4 se decidió eliminar en favor de \textbf{C++}, que pasaría a ser el único lenguaje compatible con el software. Cuándo le preguntaron acerca de este cambio, Tim Sweeney, fundador de Epic Games, respondió: ``¿Por qué no matamos a UnrealScript y pasamos a C++ puro? Ya sabes, máximo rendimiento y máxima depuración. Nos da todas estas ventajas''\cite{removeuscript}.
\item \textbf{Precio}: 5\% de royalties sobre los ingresos brutos después de los primeros \$3000 por producto y trimestre fiscal \cite{eulaunreal}.
\end{itemize}

Pese a los blueprints, que facilitan enormemente el trabajo lógico del juego, la curva de aprendizaje de Unreal Engine es algo mayor que para otros motores de juego, tal y como coinciden en múltiples foros de internet \cite{curveunreal}. El hecho de trabajar con C++, un lenguaje antiguo y de bajo nivel, ya supone un incremento de dificultad.

\subsection{GameMaker}
Otro de los motores que suele aparecer en listas entre los más populares es \textbf{GameMaker Studio 2} (en su última versión) \cite{topengines}. Su ventaja fundamental sobre la mayor parte de los motores es que no requiere de conocimientos de programación, lo que lo hace ideal para gente más comprometida con el puro diseño. Juegos condecorados por la crítica como Hyper Light Drifter y Undertale han sido desarrollados a través de este motor \cite{gamesgamemaker}.

\begin{itemize}
\item \textbf{Plataformas soportadas}: a partir de su versión 2, es posible ser usado tanto desde Windows como de Mac. Soporta exportación a Windows desktop, Mac OS X, Ubuntu, Android, iOS, fireTV, Android TV, Microsoft UWP, HTML5, PlayStation 4 y Xbox One, aunque muchas de las plataforma de exportación solo puede ser accedidas pagando \cite{getgamemaker}.
\item \textbf{Características}: múltiples herramientas para trabaja con tiles y tilesets... Sus características no parecen destacar frente a las del resto de lenguajes \cite{featuresgamemaker}. Está orientado mayormente a juegos en dos dimensiones. 
\item \textbf{Lenguajes de scripting}: aquí es donde GameMaker brilla. Y es que si no sabes cómo programar, no hay ningún problema, ya que GameMaker cuenta con un sistema llamado ``Drag and Drop'' (arrastra y suelta) que permite al desarrollador crear un juego sin escribir una sola línea de código \cite{featuresgamemaker}. Si se quiere tomar el control del juego mediante programación también es posible mediante el lenguaje propio del motor llamado GML \cite{languagegamemaker}, basado en C.
\item \textbf{Precio}: aunque el motor puede usarse y probarse de forma gratuita, es necesario pagar para publicar los juegos. En función de la plataforma elegida el precio oscila \cite{getgamemaker}.
\end{itemize}

Aunque por un lado su ``Drag and Drop'' es una característica extremadamente interesante, para nuestro propósito es más bien inútil. Incluso si quisiéramos usar su lenguaje propio, perdemos en flexibilidad (cualquier cosa que desarrollemos solo podrá ser usada para este motor) y, además, nos dificulta el trabajo, pues no se trata de un lenguaje plenamente desarrollado y, por ende, no cuenta con las herramientas que otros tantos sí.

\subsection{Godot Engine}
De entre los motores de juego propietario, aparece \textbf{Godot} con su propuesta open-source. La última versión, Godot Engine 3, apareció a finales de enero con un nuevo motor de renderizado, flujo de renderizado basado en físicas para 3D, soporte para C\#... y otra serie de novedades \cite{newgodot}.
\begin{itemize}
\item \textbf{Plataformas soportadas}: Windows, macOS, Linux, FreeBSD, OpenBSD y Haiku. Es posible exportar los juegos a iOS, Android, Windows, macOS, Linux, UWP, *BSD, Haiku e incluso a HTML \cite{godotfeatures}.
\item \textbf{Características}: creación de tus propias herramientas, una arquitectura que combina lo mejor de la renderización directa con la eficiencia de la renderización diferida, iluminación global para gráficos en tiempo real, edito de ``tile-maps'', transiciones personalizadas para crear animaciones...\cite{godotfeatures}
\item \textbf{Lenguajes de scripting}: Godot tiene un lenguaje propio, muy similar a Python, llamado GDScript. Su parecido con Python equivale a sencillez de desarrollo. Además, soporta C++ y, desde su última versión, C\# \cite{godotfeatures}. Se está trabajando asimismo en la \MYhref{https://github.com/touilleMan/godot-python}{implantación de Python} como lenguaje de scripting nativo.
\item \textbf{Precio}: quizás el punto fuerte de Godot: es totalmente gratuito, open-source y además está liberado bajo una licencia MIT, sin restricciones \cite{godotfree}.
\end{itemize}

Pese a no estar soportado de la forma en que un motor propietario de Unreal puede estarlo, Godot es un motor realmente interesante con prestaciones más que suficientes para el trabajo que nosotros pretendemos llevar a cabo.

\subsection{Unity}
\textbf{Unity}, de la compañía Unity Technologies, no es solo un motor de juego, sino además una de las competencias tecnológicas de mayor demanda a día de hoy. Cuenta con una de las tasas de crecimiento proyectado más altas, de más del 35\% en los próximos dos años \cite{learnunity}.
\begin{itemize}
\item \textbf{Plataformas soportadas}: Su editor está únicamente en Windows y macOS \cite{srunity}. En cambio, las plataformas a las que los juegos pueden ser exportados son abundantes: Windows, SteamOs, macOS, Linux, Android, iOS, Tizen, tvOS, Android TV, PlayStation 4 y Vita, Xbox One y 360, Wii U, Nintendo Switch y 3DS.
\item \textbf{Características}: incluye herramientas para el diseño de escenas en 2D y 3D, storytelling y cinemáticas, iluminación, sistema de audio, herramientas para la gestión de sprites, efectos de partículas... Todo ello junto a un motor de renderizado en tiempo real y una APl de gráficos nativos. Su editor presenta herramientas múltiples que permiten una edición rápida e intutiva \cite{featuresunity}.
\item \textbf{Lenguajes de sripting}: Unity soporta dos lenguajes nativamente: C\# y UnityScript (similar a JavaScript), aunque también pueden ser utilizados otros lenguajes de la plataorma .NET si son compilados en un \textit{.dll} gestionado compatible \cite{languagesunity}.
\item \textbf{Precio}: Unity cuenta con una versión gratuita (llamada ``personal'') que cuenta con herramientas de sobra para nuestro propósito. Otras versiones con más prestaciones requieren de pagos mensuales, pero están orientados a desarrolladores profesionales.
\end{itemize}

El uso de C\#, lenguaje de amplio recorrido y soporte, supone uno de los mayores atractivos del uso de este motor. Además, al tratarse de un proyecto tan popular a día de hoy, existe un gran cantidad de información y documentación sobre él. Será el motor a utilizar en el proyecto y se explicará con más detalle en el siguiente capítulo.