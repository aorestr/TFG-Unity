\chapter{Estado del arte}\label{chap:ArtState}

El estado del arte se define como el nivel de desarrollo de un ámbito concreto, generalmente relacionado con el mundo técnico-científico.

\section{Motores de videojuegos}
Pondría una subsección por cada motor a considerar. Y en cada uno, comentar por qué es adecuado o no: precio, licencia, curva de aprendizaje, plataformas que soporta, facilidad para añadir librerías, o lenguaje que utilice que sea más o menos apropiado, si hay mecanismos para comunicarte con software externo (como el caso del emulador)... Con media página por cada uno, creoque basta. Unreal Engine, Amazon Lumberyard, Game Maker, Godot Engine, y no sé si alguno más que sea famosete, indicando que son los más populares (ojalá haya un informe con lista de usuarios por engine...).

Creo que tanto para hablar de Unity como de GNS3, lo suyo es:

comentar las características generales en "motores de juegos" y en la de "emuladores". y luego añadir una sección quizás con los detalles de las plataformas elegidas, hablando de las características que vas a utilizar (API de GNS3, capacidad de enlazado con DLL de Unity...). Cosas que vaya a necesitar el revisor para entender el diseño o sobre todo la integración.

\subsection{Unreal Engine}

\subsection{Amazon Lumberyard}

\subsection{Game Maker}

\subsection{Godot Engine}

\subsection{Unity}


\section{Simuladores de redes}

\subsection{Simulador 1}

\subsection{Simulador 2}

\subsection{Simulador 3}

\subsection{GNS3}

\section{Nuestra elección}

\subsection{Unity}

\subsection{GNS3}
hay que comentar qué mecanismos nos interesan de GNS3. Si no se ha explicado cómo se organiza y funciona antes, aquí es el lugar.

GNS3 es usado ampliamente como método de entrenamiento para los exámenes de Cisco. Tal es así, que cuenta con su \MYhref{https://academy.gns3.com/p/the-complete-networking-fundamentals-course-your-ccna-start}{propia academia online} de cursos en los que se utiliza el simulador para estudiar, convirtiendo la formación en más accesible. Sin embargo, las imágenes de las máquinas de Cisco, aunque pueden ser encontradas fácilmente en internet, requieren de licencia para ser usadas. Siendo así se optó por hacer uso de software libre para el proyecto. 


\subsubsection{Nodos}
Una figura esquemática que muestre todos estos conceptos, es fundamental al comienzo de un capítulo o sección importante.

Cada elemento de una red está representado en GNS3 por un elemento llamado \textbf{nodo}. Estos nodos, que pueden ser desde un router a un switch, no son más que virtualizaciones de aparatos reales. De normal, estas virtualizaciones se realizan en base a imágenes de los sistemas operativos que se integran en los aparatos. Así, podemos tener varios routers distintos de Cisco montados sobre la misma estructura, permitiéndonos jugar con ellos con verdadera facilidad.

\subsubsection{Enlaces}

\subsubsection{La API de GNS3}
Una API (que será explicada más adelante)
La primera vez que aparezca un acrónimo, debes indicar cuál es su significado. De hecho, en los títulos o como primera palabra de la frase, (o en el abstract) hay que evitar las abreviaturas.

\section{Juegos docentes}
Con todo lo anterior nuestro objetivo es tal. Ya hay ejemplos



Deberían incluirse juegos que ya existan. Añade una subsección "conclusiones", o algo en la que compares qué tienen y debe incorporar tu juego, y qué cosas les falta, que vas a poner en el tuyo.