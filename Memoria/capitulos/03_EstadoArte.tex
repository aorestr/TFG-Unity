\chapter{Estado del arte}\label{chap:ArtState}

El estado del arte se define como el nivel de desarrollo de un ámbito concreto, generalmente relacionado con el mundo técnico-científico.

\section{Motores de videojuegos}

\subsection{Motores más famosos}

\subsection{Nuestra elección: Unity}

huigk

\section{Simuladores de redes}

\subsection{Simuladores más famosos}

\subsection{Nuestra elección: GNS3}

\subsubsection{Nodos}
Cada elemento de una red está representado en GNS3 por un elemento llamado \textbf{nodo}. Estos nodos, que pueden ser desde un router a un switch, no son más que virtualizaciones de aparatos reales. De normal, estas virtualizaciones se realizan en base a imágenes de los sistemas operativos que se integran en los aparatos. Así, podemos tener varios routers distintos de Cisco montados sobre la misma estructura, permitiéndonos jugar con ellos con verdadera facilidad.

\subsubsection{Enlaces}

\subsubsection{La API de GNS3}



\section{Juegos docentes}

Con todo lo anterior nuestro objetivo es tal. Ya hay ejemplos