\chapter{Estado del arte}\label{chap:ArtState}

El estado del arte se define como el nivel de desarrollo de un ámbito concreto, generalmente relacionado con el mundo técnico-científico.

\section{Simuladores de redes}

\subsection{Simulador 1}

\subsection{Simulador 2}

\subsection{Simulador 3}

\subsection{GNS3}


\section{Motores de videojuegos}
Pondría una subsección por cada motor a considerar. Y en cada uno, comentar por qué es adecuado o no: precio, licencia, curva de aprendizaje, plataformas que soporta, facilidad para añadir librerías, o lenguaje que utilice que sea más o menos apropiado, si hay mecanismos para comunicarte con software externo (como el caso del emulador)... Con media página por cada uno, creoque basta. Unreal Engine, Amazon Lumberyard, Game Maker, Godot Engine, y no sé si alguno más que sea famosete, indicando que son los más populares (ojalá haya un informe con lista de usuarios por engine...).

Creo que tanto para hablar de Unity como de GNS3, lo suyo es:

comentar las características generales en "motores de juegos" y en la de "emuladores". y luego añadir una sección quizás con los detalles de las plataformas elegidas, hablando de las características que vas a utilizar (API de GNS3, capacidad de enlazado con DLL de Unity...). Cosas que vaya a necesitar el revisor para entender el diseño o sobre todo la integración.

\subsection{Unreal Engine}

\subsection{Amazon Lumberyard}

\subsection{Game Maker}

\subsection{Godot Engine}

\subsection{Unity}

\section{Juegos docentes}
Deberían incluirse juegos que ya existan. Añade una subsección "conclusiones", o algo en la que compares qué tienen y debe incorporar tu juego, y qué cosas les falta, que vas a poner en el tuyo.