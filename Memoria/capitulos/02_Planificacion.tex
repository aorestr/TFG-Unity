\chapter{Planificación}\label{chap:Plan}

fgxdh

\section{Planificación temporal}

huigk

\section{Recursos empleados}
Dividiremos los recursos utilizados para la realización del trabajo en tres: recursos humanos, recursos software y recursos hardware:

\subsection{Recursos humanos}
Aquí una lista de las personas que han participado en el proyecto y su papel en el mismo.
\begin{itemize}
\item Juan José Ramos Muñoz, profesor del departamento de Teoría de
la Señal, Telemática y Comunicaciones en la ETSIIT de la Universidad de Granada. Tutor del proyecto
\item Jonathan Prados Garzón, profesor del departamento de Teoría de
la Señal, Telemática y Comunicaciones en la ETSIIT de la Universidad de Granada.
Cotutor del proyecto.
\item Ángel Oreste Rodríguez Romero, alumno del grado de ingeniería de tecnologías de telecomunicación en la ETSIIT de la Universidad de Granada. Autor del proyecto.
\end{itemize}

\subsection{Recursos software}
Aplicaciones y demás material digital que ha sido necesario para desarrollar el trabajo:
\begin{itemize}
\item \href{https://www.microsoft.com/es-es/windows}{\textbf{Microsoft Windows 10 Home versión 1803}} como sistema operativo sobre el que se ha llevado todo el proyecto a cabo.
\item Se usará además \href{https://xubuntu.org/about/}{\textbf{Xubuntu 18.04}} instalado en una máquina virtual para llevar ciertas pruebas del capítulo \ref{chap:Pruebas} a cabo.
\item \href{https://www.mozilla.org/es-ES/firefox/}{\textbf{Mozilla Firefox Quantum}} como navegador web. 61.0.2 es la última versión empleada.
\item Para la realización de la memoria:
\begin{itemize}
\item El editor \href{http://www.xm1math.net/texmaker/}{\textbf{Texmaker 5.0.2}} junto a \href{https://miktex.org/about}{MiKTeX 2.9}, implementación de \LaTeX.
\item \href{https://es.libreoffice.org/descubre/draw/}{\textbf{LibreOffice Draw 6.1.0.3}} para el diseño de algunos esquemas.
\item \href{https://inkscape.org/es/}{\textbf{InkScape 0.92}} para la transformación de imágenes vectoriales en archivos de formato \textit{.pdf}.
\end{itemize}
\item Como lector de archivos \textit{.pdf} ha sido elegido \href{https://www.foxitsoftware.com/pdf-reader/}{\textbf{Foxit Reader}}. Última versión: 9.2.0.
\item Para el desarrollo de código:
\begin{itemize}
\item El editor más utilizado durante el transcurso del trabajo ha sido \href{https://code.visualstudio.com/}{\textbf{Visual Studio Code}}. Ha pasado por varias versiones durante su uso. La más reciente es 1.26.
\item Para la compilación de código y el scripting en Unity el hermano mayor del anterior, \href{https://visualstudio.microsoft.com/es/}{\textbf{Microsoft Visual Studio}}, también ha pasado por varias versiones, pero la última de ellas ha sido la 15.8.0.
\item \href{https://notepad-plus-plus.org/}{\textbf{Notepad++}} ha sido utilizado para tomar notas y revisar código de forma rápida y ligera. La última versión usada es la 7.5.8.
\item \href{https://www.visual-paradigm.com/}{Visual Paradigm Enterprise 15.1} para dibujar los diagramas UML.
\end{itemize}
\item Para la virtualización de redes:
\begin{itemize}
\item \href{https://www.gns3.com/}{\textbf{GNS3 2.1.3 y finalmente 2.1.9}} como simulador de redes.
\item \href{https://www.virtualbox.org/}{\textbf{VirtualBox 5.2.16}} para la virtualización de ciertos dispositivos de la red y la instalación de un sistema Linux (Xubuntu) en el que integrar GNS3.
\end{itemize}
\item El motor para el desarrollo de videojuegos \href{https://unity3d.com/es}{\textbf{Unity}} versión personal. El último número de versión empleado ha sido el 2018.2.2f1.
\item Para el capítulo \ref{chap:Pruebas} serán útil href{https://docs.microsoft.com/en-us/sysinternals/downloads/rammap}{RamMap 1.51} y href{https://docs.microsoft.com/en-us/sysinternals/downloads/process-explorer}{Process Explorer 16.21}.
\end{itemize}

\subsection{Recursos hardware}
Todo el proyecto ha sido construido a través del portátil Lenovo Z500. Cuenta con un procesador Intel i7 3232QM de ocho núcleos virtuales, 16GB de RAM DDR3 a 798MHz, un disco duro HDD de 1GB a 5400RPM y una tarjeta gráfica dedicada Nvidia 740M con 1GB VRAM.

\section{Presupuesto}



