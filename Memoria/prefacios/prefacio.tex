\chapter*{}
%\thispagestyle{empty}
%\cleardoublepage

%\thispagestyle{empty}

\begin{titlepage}
 
 
\setlength{\centeroffset}{-0.5\oddsidemargin}
\addtolength{\centeroffset}{0.5\evensidemargin}
\thispagestyle{empty}

\noindent\hspace*{\centeroffset}\begin{minipage}{\textwidth}

\centering
%\includegraphics[width=0.9\textwidth]{imagenes/logo_ugr.jpg}\\[1.4cm]



 \vspace{3.3cm}

%si el proyecto tiene logo poner aquí
\includegraphics[scale=0.5]{imagenes/logo.png} 
 \vspace{0.5cm}

% Title

{\Huge\bfseries Desarrollo de un videojuego para la configuración y análisis de redes de computadores\\
}
\noindent\rule[-1ex]{\textwidth}{3pt}\\[3.5ex]
{\large\bfseries GNS3sharp\\[4cm]}
\end{minipage}

\vspace{2.5cm}
\noindent\hspace*{\centeroffset}\begin{minipage}{\textwidth}
\centering

\textbf{Autor}\\ {Ángel Oreste Rodríguez Romero}\\[2.5ex]
\textbf{Directores}\\
{Juan José Ramos Muñoz\\
Jonathan Prados Garzón}\\[2cm]
%\includegraphics[width=0.15\textwidth]{imagenes/tstc.png}\\[0.1cm]
%\textsc{Departamento de Teoría de la Señal, Telemática y Comunicaciones}\\
%\textsc{---}\\
%Granada, mes de 201
\end{minipage}
%\addtolength{\textwidth}{\centeroffset}
\vspace{\stretch{2}}

 
\end{titlepage}






\cleardoublepage
\thispagestyle{empty}

\begin{center}
{\large\bfseries Desarrollo de un videojuego para la configuración y análisis de redes de computadores: GNS3sharp}\\
\end{center}
\begin{center}
Ángel Oreste Rodríguez Romero\\
\end{center}

%\vspace{0.7cm}
\noindent{\textbf{Palabras clave}: palabra\_clave1, palabra\_clave2, palabra\_clave3, ......}\\

\vspace{0.7cm}
\noindent{\textbf{Resumen}}\\

El jugar ha sido desde siempre un gran amigo de la educación. Aprender jugando es un lema que cada vez se repite más. Los videojuegos, concretamente, toman en cierta forma el relevo de los juegos tal y como tradicionalmente estos se han entendido y amplían sus posibilidades.

La era digital afecta a casi todos los ámbitos de nuestro entorno. Las redes no iban a quedar excluidas de ese avance. Así, se pueden encontrar decenas de implementaciones virtuales de redes de telecomunicaciones, permitiéndonos visualizar su funcionamiento evitando el desembolso que equivale una real.

Digitalizados ambos ámbitos, ¿por qué no unirlos? ¿Y por qué no unirlos con un propósito educacional?

El presente documento pretende realizar un acercamiento a tal propósito. Se listará una serie de tecnologías que permiten llevar esto a cabo así como el desarrollo de mi aproximación.
\cleardoublepage


\thispagestyle{empty}


\begin{center}
{\large\bfseries Development of a videogame for the configuration and analysis of computer networks: GNS3sharp}\\
\end{center}
\begin{center}
Ángel Oreste, Rodríguez Romero\\
\end{center}

%\vspace{0.7cm}
\noindent{\textbf{Keywords}: Keyword1, Keyword2, Keyword3, ....}\\

\vspace{0.7cm}
\noindent{\textbf{Abstract}}\\

Playing has always been a great friend of education. Learning by playing is a motto that is repeated more and more. Videogames, in particular, take over from games as they have traditionally been understood and expand their possibilities.

The digital age affects almost every area of our environment. Networks would not be excluded from this development. Thus, dozens of virtual implementations of telecommunication networks can be found, allowing us to visualize them working, avoiding the disbursement that is equivalent to a real one.

Digitized both areas, why not unite them, and why not unite them for an educational purpose?

This document is intended to bring this about. A number of technologies will be listed that allow this to be done as well as the development of my approach.

\chapter*{}
\thispagestyle{empty}

\noindent\rule[-1ex]{\textwidth}{2pt}\\[4.5ex]

Yo, \textbf{Ángel Oreste Rodríguez Romero}, alumno de la titulación Ingeniería de Tecnologías de Telecomunicación de la \textbf{Escuela Técnica Superior
de Ingenierías Informática y de Telecomunicación de la Universidad de Granada}, con DNI 25351379C, autorizo la
ubicación de la siguiente copia de mi Trabajo Fin de Grado en la biblioteca del centro para que pueda ser
consultada por las personas que lo deseen.

\vspace{6cm}

\noindent Fdo: Ángel Oreste Rodríguez Romero

\vspace{2cm}

\begin{flushright}
Granada a 1 de septiembre de 2018.
\end{flushright}


\chapter*{}
\thispagestyle{empty}

\noindent\rule[-1ex]{\textwidth}{2pt}\\[4.5ex]

D. \textbf{Juan José Ramos Muñoz}, Profesor del Área de Telemática del Departamento TSTC de la Universidad de Granada.

\vspace{0.5cm}

D. \textbf{Jonathan Prados Garzón}, Profesor del Área de Telemática del Departamento TSTC de la Universidad de Granada.


\vspace{0.5cm}

\textbf{Informan:}

\vspace{0.5cm}

Que el presente trabajo, titulado \textit{\textbf{Desarrollo de un videojuego para la configuración y análisis de redes de computadores: GNS3sharp}},
ha sido realizado bajo su supervisión por \textbf{Ángel Oreste Rodríguez Romero}, y autorizamos la defensa de dicho trabajo ante el tribunal
que corresponda.

\vspace{0.5cm}

Y para que conste, expiden y firman el presente informe en Granada a 1 de septiembre de 2018.

\vspace{1cm}

\textbf{Los directores:}

\vspace{5cm}

\noindent \textbf{Juan José Ramos Muñoz \ \ \ \ \ Jonathan Prados Garzón}

\chapter*{Agradecimientos}
\thispagestyle{empty}

       \vspace{1cm}

A Juanjo, por su increíble paciencia e inestimable ayuda, tanto en nuestras tutorías presenciales como aquellas improvisadas por Telegram. A todos aquellos profesores que confiaron en mí y en mis capacidades más que yo mismo en tantos momentos. A todos aquellos compañeros como Alfonso que no solo me facilitaron la vida académica con su conocimiento, si no también por su compañía. A Alberto, que aunque algo reticente de primeras, está dispuesto a echarme una mano de pedírselo. A mis compañeros de Francia, que me impulsaron a ampliar mis conocimientos. A Antonio por el logo tan genial que ha hecho. A mi grupo, porque sin él no habría tenido el ánimo suficiente durante este año para afrontar el proyecto. A todos aquellos amigos que me oyeron quejarme de mi proyecto con estoicismo. A la comunidad de StackOverflow que de tantos apuros me ha sacado.

Pero ante todo, a mis padres, pilar fundamental y soporte absoluto de toda mi vida, universitaria o no. Por la educación que me han dado, por todos aquellos caprichos que me permitieron y por velar siempre por mi salud.