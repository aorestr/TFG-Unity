\chapter*{}
%\thispagestyle{empty}
%\cleardoublepage

%\thispagestyle{empty}

\input{portada/portada_2}



\cleardoublepage
\thispagestyle{empty}

\begin{center}
{\large\bfseries Desarrollo de un videojuego para la configuración y análisis de redes de computadores: GNS3sharp}\\
\end{center}
\begin{center}
Ángel Oreste Rodríguez Romero\\
\end{center}

%\vspace{0.7cm}
\noindent{\textbf{Palabras clave}: emulador, GNS3, virtualización, videojuego, API}\\

\vspace{0.7cm}
\noindent{\textbf{Resumen}}\\

En la época actual, la educación ha visto ampliadas sus fronteras hacia lo digital. Lo virtual ha propiciado un nuevo campo de entrenamiento para aprendices, que encuentran en ello una forma nueva de acceder al conocimiento. Sus posibilidades son muy amplias.

Los videojuegos, a su vez, han gozado de un crecimiento enorme en la actualidad. Además de sus evidentes usos lúdicos, estudios confirman su utilidad como herramienta de enseñanza.

Este trabajo se propone diseñar una plataforma para el desarrollo de juegos para la enseñanza de conceptos de redes de computadores. El juego posee integración con una red virtualizada, logrando así un laboratorio virtual más rico y lleno de posibilidades. Como resultado, es desarrollada una biblioteca con herramientas que permiten la integración de estas redes virtualizadas en el juego. Asimismo, un juego que muestra el funcionamiento básico de la biblioteca es creado.

Finalmente, se evalúan las posibles alternativas para el despliegue del juego, de forma que este sea cómodo y asequible para todos aquellos que se propongan jugarlo.

\cleardoublepage


\thispagestyle{empty}


\begin{center}
{\large\bfseries Development of a videogame for the configuration and analysis of computer networks: GNS3sharp}\\
\end{center}
\begin{center}
Ángel Oreste, Rodríguez Romero\\
\end{center}

%\vspace{0.7cm}
\noindent{\textbf{Keywords}: emulator, GNS3, virtualization, videogame, API}\\

\vspace{0.7cm}
\noindent{\textbf{Abstract}}\\

In the current era, education has seen its frontiers extended to the digital. Virtual world has provided a new training ground for apprentices, who find in it a new way of accessing knowledge. Its possibilities are very wide,

Videogames, in turn, have enjoyed an enormous growth at present. In addition to their obvious recreational uses, studies confirm their usefulness as a teaching tool.

This work proposes to design a platform for the development of games for the teaching of computer networks concepts. The game has integration with a virtualized network, achieving this way a richer virtual laboratory and full of possibilities. As a result, a library is developed with tools that allow the integration of these virtualized networks in the game. Also, a game that shows the basic operation of the library is created.

Finally, the possible alternatives for the deployment of the game are evaluated, so that it is comfortable and affordable for all those who intend to play it.

\chapter*{}
\thispagestyle{empty}

\noindent\rule[-1ex]{\textwidth}{2pt}\\[4.5ex]

Yo, \textbf{Ángel Oreste Rodríguez Romero}, alumno de la titulación Ingeniería de Tecnologías de Telecomunicación de la \textbf{Escuela Técnica Superior
de Ingenierías Informática y de Telecomunicación de la Universidad de Granada}, con DNI 25351379C, autorizo la
ubicación de la siguiente copia de mi Trabajo Fin de Grado en la biblioteca del centro para que pueda ser
consultada por las personas que lo deseen.

\vspace{6cm}

\noindent Fdo: Ángel Oreste Rodríguez Romero

\vspace{2cm}

\begin{flushright}
Granada a 1 de septiembre de 2018.
\end{flushright}


\chapter*{}
\thispagestyle{empty}

\noindent\rule[-1ex]{\textwidth}{2pt}\\[4.5ex]

D. \textbf{Juan José Ramos Muñoz}, Profesor del Área de Telemática del Departamento TSTC de la Universidad de Granada.

\vspace{0.5cm}

D. \textbf{Jonathan Prados Garzón}, Profesor del Área de Telemática del Departamento TSTC de la Universidad de Granada.


\vspace{0.5cm}

\textbf{Informan:}

\vspace{0.5cm}

Que el presente trabajo, titulado \textit{\textbf{Desarrollo de un videojuego para la configuración y análisis de redes de computadores: GNS3sharp}},
ha sido realizado bajo su supervisión por \textbf{Ángel Oreste Rodríguez Romero}, y autorizamos la defensa de dicho trabajo ante el tribunal
que corresponda.

\vspace{0.5cm}

Y para que conste, expiden y firman el presente informe en Granada a 1 de septiembre de 2018.

\vspace{1cm}

\textbf{Los directores:}

\vspace{5cm}

\noindent \textbf{Juan José Ramos Muñoz \ \ \ \ \ Jonathan Prados Garzón}

\chapter*{Agradecimientos}
\thispagestyle{empty}

       \vspace{1cm}

A Juanjo, por su increíble paciencia e inestimable ayuda, tanto en nuestras tutorías presenciales como aquellas improvisadas por Telegram. A todos esos profesores que confiaron en mí y en mis capacidades más que yo mismo en tantos momentos. A todos aquellos compañeros como Alfonso que no solo me facilitaron la vida académica con su conocimiento, sino también con su compañía. A Alberto, que aunque algo reticente de primeras, está dispuesto a echarme una mano de pedírselo. A mis compañeros de Francia, que me impulsaron a ampliar mis conocimientos. A Antonio por el logo tan genial que ha hecho. A mi grupo, porque sin él no habría tenido el ánimo suficiente durante este año para afrontar el proyecto. A todos aquellos amigos que me oyeron quejarme de mi proyecto con estoicismo. A la comunidad de StackOverflow que de tantos apuros me ha sacado.

Pero ante todo, a mis padres, pilar fundamental y soporte absoluto de toda mi vida, universitaria o no. Por la educación que me han dado, por todos aquellos caprichos que me permitieron y por velar siempre por mi salud.